\documentclass{beamer}


\usepackage[orientation=landscape,size=a0,scale=1.4,debug]{beamerposter}
\mode<presentation>{\usetheme{mlr}}


\usepackage[utf8]{inputenc} % UTF-8
\usepackage[english]{babel} % Language
\usepackage{hyperref} % Hyperlinks
\usepackage{ragged2e} % Text position
\usepackage[export]{adjustbox} % Image position
\usepackage[most]{tcolorbox}
\usepackage{amsmath}
\usepackage{mathtools}
\usepackage{dsfont}
\usepackage{verbatim}
\usepackage{amsmath}
\usepackage{amsfonts}
\usepackage{csquotes}
\usepackage{multirow}
\usepackage{longtable}
\usepackage{enumerate}
\usepackage[absolute,overlay]{textpos}
\usepackage{psfrag}
\usepackage{algorithm}
\usepackage{algpseudocode}
\usepackage{eqnarray}
\usepackage{arydshln}
\usepackage{tabularx}
\usepackage{placeins}
\usepackage{tikz}
\usepackage{setspace}
\usepackage{colortbl}
\usepackage{mathtools}
\usepackage{wrapfig}
\usepackage{bm}
\usepackage{nicefrac}

% math spaces
\ifdefined\N                                                                
\renewcommand{\N}{\mathds{N}} % N, naturals
\else \newcommand{\N}{\mathds{N}} \fi 
\newcommand{\Z}{\mathds{Z}} % Z, integers
\newcommand{\Q}{\mathds{Q}} % Q, rationals
\newcommand{\R}{\mathds{R}} % R, reals
\ifdefined\C 
  \renewcommand{\C}{\mathds{C}} % C, complex
\else \newcommand{\C}{\mathds{C}} \fi
\newcommand{\continuous}{\mathcal{C}} % C, space of continuous functions
\newcommand{\M}{\mathcal{M}} % machine numbers
\newcommand{\epsm}{\epsilon_m} % maximum error

% counting / finite sets
\newcommand{\setzo}{\{0, 1\}} % set 0, 1
\newcommand{\setmp}{\{-1, +1\}} % set -1, 1
\newcommand{\unitint}{[0, 1]} % unit interval

% basic math stuff
\newcommand{\xt}{\tilde x} % x tilde
\newcommand{\argmax}{\operatorname{arg\,max}} % argmax
\newcommand{\argmin}{\operatorname{arg\,min}} % argmin
\newcommand{\argminlim}{\mathop{\mathrm{arg\,min}}\limits} % argmax with limits
\newcommand{\argmaxlim}{\mathop{\mathrm{arg\,max}}\limits} % argmin with limits  
\newcommand{\sign}{\operatorname{sign}} % sign, signum
\newcommand{\I}{\mathbb{I}} % I, indicator
\newcommand{\order}{\mathcal{O}} % O, order
\newcommand{\pd}[2]{\frac{\partial{#1}}{\partial #2}} % partial derivative
\newcommand{\floorlr}[1]{\left\lfloor #1 \right\rfloor} % floor
\newcommand{\ceillr}[1]{\left\lceil #1 \right\rceil} % ceiling

% sums and products
\newcommand{\sumin}{\sum\limits_{i=1}^n} % summation from i=1 to n
\newcommand{\sumim}{\sum\limits_{i=1}^m} % summation from i=1 to m
\newcommand{\sumjn}{\sum\limits_{j=1}^n} % summation from j=1 to p
\newcommand{\sumjp}{\sum\limits_{j=1}^p} % summation from j=1 to p
\newcommand{\sumik}{\sum\limits_{i=1}^k} % summation from i=1 to k
\newcommand{\sumkg}{\sum\limits_{k=1}^g} % summation from k=1 to g
\newcommand{\sumjg}{\sum\limits_{j=1}^g} % summation from j=1 to g
\newcommand{\meanin}{\frac{1}{n} \sum\limits_{i=1}^n} % mean from i=1 to n
\newcommand{\meanim}{\frac{1}{m} \sum\limits_{i=1}^m} % mean from i=1 to n
\newcommand{\meankg}{\frac{1}{g} \sum\limits_{k=1}^g} % mean from k=1 to g
\newcommand{\prodin}{\prod\limits_{i=1}^n} % product from i=1 to n
\newcommand{\prodkg}{\prod\limits_{k=1}^g} % product from k=1 to g
\newcommand{\prodjp}{\prod\limits_{j=1}^p} % product from j=1 to p

% linear algebra
\newcommand{\one}{\boldsymbol{1}} % 1, unitvector
\newcommand{\zero}{\mathbf{0}} % 0-vector
\newcommand{\id}{\boldsymbol{I}} % I, identity
\newcommand{\diag}{\operatorname{diag}} % diag, diagonal
\newcommand{\trace}{\operatorname{tr}} % tr, trace
\newcommand{\spn}{\operatorname{span}} % span
\newcommand{\scp}[2]{\left\langle #1, #2 \right\rangle} % <.,.>, scalarproduct
\newcommand{\mat}[1]{\begin{pmatrix} #1 \end{pmatrix}} % short pmatrix command
\newcommand{\Amat}{\mathbf{A}} % matrix A
\newcommand{\Deltab}{\mathbf{\Delta}} % error term for vectors

% basic probability + stats
\renewcommand{\P}{\mathds{P}} % P, probability
\newcommand{\E}{\mathds{E}} % E, expectation
\newcommand{\var}{\mathsf{Var}} % Var, variance
\newcommand{\cov}{\mathsf{Cov}} % Cov, covariance
\newcommand{\corr}{\mathsf{Corr}} % Corr, correlation
\newcommand{\normal}{\mathcal{N}} % N of the normal distribution
\newcommand{\iid}{\overset{i.i.d}{\sim}} % dist with i.i.d superscript
\newcommand{\distas}[1]{\overset{#1}{\sim}} % ... is distributed as ...

% machine learning
\newcommand{\Xspace}{\mathcal{X}} % X, input space
\newcommand{\Yspace}{\mathcal{Y}} % Y, output space
\newcommand{\nset}{\{1, \ldots, n\}} % set from 1 to n
\newcommand{\pset}{\{1, \ldots, p\}} % set from 1 to p
\newcommand{\gset}{\{1, \ldots, g\}} % set from 1 to g
\newcommand{\Pxy}{\mathbb{P}_{xy}} % P_xy
\newcommand{\Exy}{\mathbb{E}_{xy}} % E_xy: Expectation over random variables xy
\newcommand{\xv}{\mathbf{x}} % vector x (bold)
\newcommand{\xtil}{\tilde{\mathbf{x}}} % vector x-tilde (bold)
\newcommand{\yv}{\mathbf{y}} % vector y (bold)
\newcommand{\xy}{(\xv, y)} % observation (x, y)
\newcommand{\xvec}{\left(x_1, \ldots, x_p\right)^\top} % (x1, ..., xp) 
\newcommand{\Xmat}{\mathbf{X}} % Design matrix
\newcommand{\allDatasets}{\mathds{D}} % The set of all datasets
\newcommand{\allDatasetsn}{\mathds{D}_n}  % The set of all datasets of size n 
\newcommand{\D}{\mathcal{D}} % D, data
\newcommand{\Dn}{\D_n} % D_n, data of size n
\newcommand{\Dtrain}{\mathcal{D}_{\text{train}}} % D_train, training set
\newcommand{\Dtest}{\mathcal{D}_{\text{test}}} % D_test, test set
\newcommand{\xyi}[1][i]{\left(\xv^{(#1)}, y^{(#1)}\right)} % (x^i, y^i), i-th observation
\newcommand{\Dset}{\left( \xyi[1], \ldots, \xyi[n]\right)} % {(x1,y1)), ..., (xn,yn)}, data
\newcommand{\defAllDatasetsn}{(\Xspace \times \Yspace)^n} % Def. of the set of all datasets of size n 
\newcommand{\defAllDatasets}{\bigcup_{n \in \N}(\Xspace \times \Yspace)^n} % Def. of the set of all datasets 
\newcommand{\xdat}{\left\{ \xv^{(1)}, \ldots, \xv^{(n)}\right\}} % {x1, ..., xn}, input data
\newcommand{\ydat}{\left\{ \yv^{(1)}, \ldots, \yv^{(n)}\right\}} % {y1, ..., yn}, input data
\newcommand{\yvec}{\left(y^{(1)}, \hdots, y^{(n)}\right)^\top} % (y1, ..., yn), vector of outcomes
\renewcommand{\xi}[1][i]{\xv^{(#1)}} % x^i, i-th observed value of x
\newcommand{\yi}[1][i]{y^{(#1)}} % y^i, i-th observed value of y 
\newcommand{\xivec}{\left(x^{(i)}_1, \ldots, x^{(i)}_p\right)^\top} % (x1^i, ..., xp^i), i-th observation vector
\newcommand{\xj}{\xv_j} % x_j, j-th feature
\newcommand{\xjvec}{\left(x^{(1)}_j, \ldots, x^{(n)}_j\right)^\top} % (x^1_j, ..., x^n_j), j-th feature vector
\newcommand{\phiv}{\mathbf{\phi}} % Basis transformation function phi
\newcommand{\phixi}{\mathbf{\phi}^{(i)}} % Basis transformation of xi: phi^i := phi(xi)

%%%%%% ml - models general
\newcommand{\lamv}{\bm{\lambda}} % lambda vector, hyperconfiguration vector
\newcommand{\Lam}{\bm{\Lambda}}	 % Lambda, space of all hpos
% Inducer / Inducing algorithm
\newcommand{\preimageInducer}{\left(\defAllDatasets\right)\times\Lam} % Set of all datasets times the hyperparameter space
\newcommand{\preimageInducerShort}{\allDatasets\times\Lam} % Set of all datasets times the hyperparameter space
% Inducer / Inducing algorithm
\newcommand{\ind}{\mathcal{I}} % Inducer, inducing algorithm, learning algorithm 

% continuous prediction function f
\newcommand{\ftrue}{f_{\text{true}}}  % True underlying function (if a statistical model is assumed)
\newcommand{\ftruex}{\ftrue(\xv)} % True underlying function (if a statistical model is assumed)
\newcommand{\fx}{f(\xv)} % f(x), continuous prediction function
\newcommand{\fdomains}{f: \Xspace \rightarrow \R^g} % f with domain and co-domain
\newcommand{\Hspace}{\mathcal{H}} % hypothesis space where f is from
\newcommand{\fbayes}{f^{\ast}} % Bayes-optimal model
\newcommand{\fxbayes}{f^{\ast}(\xv)} % Bayes-optimal model
\newcommand{\fkx}[1][k]{f_{#1}(\xv)} % f_j(x), discriminant component function
\newcommand{\fh}{\hat{f}} % f hat, estimated prediction function
\newcommand{\fxh}{\fh(\xv)} % fhat(x)
\newcommand{\fxt}{f(\xv ~|~ \thetab)} % f(x | theta)
\newcommand{\fxi}{f\left(\xv^{(i)}\right)} % f(x^(i))
\newcommand{\fxih}{\hat{f}\left(\xv^{(i)}\right)} % f(x^(i))
\newcommand{\fxit}{f\left(\xv^{(i)} ~|~ \thetab\right)} % f(x^(i) | theta)
\newcommand{\fhD}{\fh_{\D}} % fhat_D, estimate of f based on D
\newcommand{\fhDtrain}{\fh_{\Dtrain}} % fhat_Dtrain, estimate of f based on D
\newcommand{\fhDnlam}{\fh_{\Dn, \lamv}} %model learned on Dn with hp lambda
\newcommand{\fhDlam}{\fh_{\D, \lamv}} %model learned on D with hp lambda
\newcommand{\fhDnlams}{\fh_{\Dn, \lamv^\ast}} %model learned on Dn with optimal hp lambda 
\newcommand{\fhDlams}{\fh_{\D, \lamv^\ast}} %model learned on D with optimal hp lambda 

% discrete prediction function h
\newcommand{\hx}{h(\xv)} % h(x), discrete prediction function
\newcommand{\hh}{\hat{h}} % h hat
\newcommand{\hxh}{\hat{h}(\xv)} % hhat(x)
\newcommand{\hxt}{h(\xv | \thetab)} % h(x | theta)
\newcommand{\hxi}{h\left(\xi\right)} % h(x^(i))
\newcommand{\hxit}{h\left(\xi ~|~ \thetab\right)} % h(x^(i) | theta)
\newcommand{\hbayes}{h^{\ast}} % Bayes-optimal classification model
\newcommand{\hxbayes}{h^{\ast}(\xv)} % Bayes-optimal classification model

% yhat
\newcommand{\yh}{\hat{y}} % yhat for prediction of target
\newcommand{\yih}{\hat{y}^{(i)}} % yhat^(i) for prediction of ith targiet
\newcommand{\resi}{\yi- \yih}

% theta
\newcommand{\thetah}{\hat{\theta}} % theta hat
\newcommand{\thetab}{\bm{\theta}} % theta vector
\newcommand{\thetabh}{\bm{\hat\theta}} % theta vector hat
\newcommand{\thetat}[1][t]{\thetab^{[#1]}} % theta^[t] in optimization
\newcommand{\thetatn}[1][t]{\thetab^{[#1 +1]}} % theta^[t+1] in optimization
\newcommand{\thetahDnlam}{\thetabh_{\Dn, \lamv}} %theta learned on Dn with hp lambda
\newcommand{\thetahDlam}{\thetabh_{\D, \lamv}} %theta learned on D with hp lambda
\newcommand{\mint}{\min_{\thetab \in \Theta}} % min problem theta
\newcommand{\argmint}{\argmin_{\thetab \in \Theta}} % argmin theta

% densities + probabilities
% pdf of x 
\newcommand{\pdf}{p} % p
\newcommand{\pdfx}{p(\xv)} % p(x)
\newcommand{\pixt}{\pi(\xv~|~ \thetab)} % pi(x|theta), pdf of x given theta
\newcommand{\pixit}[1][i]{\pi\left(\xi[#1] ~|~ \thetab\right)} % pi(x^i|theta), pdf of x given theta
\newcommand{\pixii}{\pi\left(\xi\right)} % pi(x^i), pdf of i-th x 

% pdf of (x, y)
\newcommand{\pdfxy}{p(\xv,y)} % p(x, y)
\newcommand{\pdfxyt}{p(\xv, y ~|~ \thetab)} % p(x, y | theta)
\newcommand{\pdfxyit}{p\left(\xi, \yi ~|~ \thetab\right)} % p(x^(i), y^(i) | theta)

% pdf of x given y
\newcommand{\pdfxyk}[1][k]{p(\xv | y= #1)} % p(x | y = k)
\newcommand{\lpdfxyk}[1][k]{\log p(\xv | y= #1)} % log p(x | y = k)
\newcommand{\pdfxiyk}[1][k]{p\left(\xi | y= #1 \right)} % p(x^i | y = k)

% prior probabilities
\newcommand{\pik}[1][k]{\pi_{#1}} % pi_k, prior
\newcommand{\lpik}[1][k]{\log \pi_{#1}} % log pi_k, log of the prior
\newcommand{\pit}{\pi(\thetab)} % Prior probability of parameter theta

% posterior probabilities
\newcommand{\post}{\P(y = 1 ~|~ \xv)} % P(y = 1 | x), post. prob for y=1
\newcommand{\postk}[1][k]{\P(y = #1 ~|~ \xv)} % P(y = k | y), post. prob for y=k
\newcommand{\pidomains}{\pi: \Xspace \rightarrow \unitint} % pi with domain and co-domain
\newcommand{\pibayes}{\pi^{\ast}} % Bayes-optimal classification model
\newcommand{\pixbayes}{\pi^{\ast}(\xv)} % Bayes-optimal classification model
\newcommand{\pix}{\pi(\xv)} % pi(x), P(y = 1 | x)
\newcommand{\piv}{\bm{\pi}} % pi, bold, as vector
\newcommand{\pikx}[1][k]{\pi_{#1}(\xv)} % pi_k(x), P(y = k | x)
\newcommand{\pikxt}[1][k]{\pi_{#1}(\xv ~|~ \thetab)} % pi_k(x | theta), P(y = k | x, theta)
\newcommand{\pixh}{\hat \pi(\xv)} % pi(x) hat, P(y = 1 | x) hat
\newcommand{\pikxh}[1][k]{\hat \pi_{#1}(\xv)} % pi_k(x) hat, P(y = k | x) hat
\newcommand{\pixih}{\hat \pi(\xi)} % pi(x^(i)) with hat
\newcommand{\pikxih}[1][k]{\hat \pi_{#1}(\xi)} % pi_k(x^(i)) with hat
\newcommand{\pdfygxt}{p(y ~|~\xv, \thetab)} % p(y | x, theta)
\newcommand{\pdfyigxit}{p\left(\yi ~|~\xi, \thetab\right)} % p(y^i |x^i, theta)
\newcommand{\lpdfygxt}{\log \pdfygxt } % log p(y | x, theta)
\newcommand{\lpdfyigxit}{\log \pdfyigxit} % log p(y^i |x^i, theta)

% probababilistic
\newcommand{\bayesrulek}[1][k]{\frac{\P(\xv | y= #1) \P(y= #1)}{\P(\xv)}} % Bayes rule
\newcommand{\muk}{\bm{\mu_k}} % mean vector of class-k Gaussian (discr analysis) 

% residual and margin
\newcommand{\eps}{\epsilon} % residual, stochastic
\newcommand{\epsi}{\epsilon^{(i)}} % epsilon^i, residual, stochastic
\newcommand{\epsh}{\hat{\epsilon}} % residual, estimated
\newcommand{\yf}{y \fx} % y f(x), margin
\newcommand{\yfi}{\yi \fxi} % y^i f(x^i), margin
\newcommand{\Sigmah}{\hat \Sigma} % estimated covariance matrix
\newcommand{\Sigmahj}{\hat \Sigma_j} % estimated covariance matrix for the j-th class

% ml - loss, risk, likelihood
\newcommand{\Lyf}{L\left(y, f\right)} % L(y, f), loss function
\newcommand{\Lypi}{L\left(y, \pi\right)} % L(y, pi), loss function
\newcommand{\Lxy}{L\left(y, \fx\right)} % L(y, f(x)), loss function
\newcommand{\Lxyi}{L\left(\yi, \fxi\right)} % loss of observation
\newcommand{\Lxyt}{L\left(y, \fxt\right)} % loss with f parameterized
\newcommand{\Lxyit}{L\left(\yi, \fxit\right)} % loss of observation with f parameterized
\newcommand{\Lxym}{L\left(\yi, f\left(\bm{\tilde{x}}^{(i)} ~|~ \thetab\right)\right)} % loss of observation with f parameterized
\newcommand{\Lpixy}{L\left(y, \pix\right)} % loss in classification
\newcommand{\Lpiv}{L\left(y, \piv\right)} % loss in classification
\newcommand{\Lpixyi}{L\left(\yi, \pixii\right)} % loss of observation in classification
\newcommand{\Lpixyt}{L\left(y, \pixt\right)} % loss with pi parameterized
\newcommand{\Lpixyit}{L\left(\yi, \pixit\right)} % loss of observation with pi parameterized
\newcommand{\Lhxy}{L\left(y, \hx\right)} % L(y, h(x)), loss function on discrete classes
\newcommand{\Lr}{L\left(r\right)} % L(r), loss defined on residual (reg) / margin (classif)
\newcommand{\lone}{|y - \fx|} % L1 loss
\newcommand{\ltwo}{\left(y - \fx\right)^2} % L2 loss
\newcommand{\lbernoullimp}{\ln(1 + \exp(-y \cdot \fx))} % Bernoulli loss for -1, +1 encoding
\newcommand{\lbernoullizo}{- y \cdot \fx + \log(1 + \exp(\fx))} % Bernoulli loss for 0, 1 encoding
\newcommand{\lcrossent}{- y \log \left(\pix\right) - (1 - y) \log \left(1 - \pix\right)} % cross-entropy loss
\newcommand{\lbrier}{\left(\pix - y \right)^2} % Brier score
\newcommand{\risk}{\mathcal{R}} % R, risk
\newcommand{\riskbayes}{\mathcal{R}^\ast}
\newcommand{\riskf}{\risk(f)} % R(f), risk
\newcommand{\riskdef}{\E_{y|\xv}\left(\Lxy \right)} % risk def (expected loss)
\newcommand{\riskt}{\mathcal{R}(\thetab)} % R(theta), risk
\newcommand{\riske}{\mathcal{R}_{\text{emp}}} % R_emp, empirical risk w/o factor 1 / n
\newcommand{\riskeb}{\bar{\mathcal{R}}_{\text{emp}}} % R_emp, empirical risk w/ factor 1 / n
\newcommand{\riskef}{\riske(f)} % R_emp(f)
\newcommand{\risket}{\mathcal{R}_{\text{emp}}(\thetab)} % R_emp(theta)
\newcommand{\riskr}{\mathcal{R}_{\text{reg}}} % R_reg, regularized risk
\newcommand{\riskrt}{\mathcal{R}_{\text{reg}}(\thetab)} % R_reg(theta)
\newcommand{\riskrf}{\riskr(f)} % R_reg(f)
\newcommand{\riskrth}{\hat{\mathcal{R}}_{\text{reg}}(\thetab)} % hat R_reg(theta)
\newcommand{\risketh}{\hat{\mathcal{R}}_{\text{emp}}(\thetab)} % hat R_emp(theta)
\newcommand{\LL}{\mathcal{L}} % L, likelihood
\newcommand{\LLt}{\mathcal{L}(\thetab)} % L(theta), likelihood
\newcommand{\LLtx}{\mathcal{L}(\thetab | \xv)} % L(theta|x), likelihood
\newcommand{\logl}{\ell} % l, log-likelihood
\newcommand{\loglt}{\logl(\thetab)} % l(theta), log-likelihood
\newcommand{\logltx}{\logl(\thetab | \xv)} % l(theta|x), log-likelihood
\newcommand{\errtrain}{\text{err}_{\text{train}}} % training error
\newcommand{\errtest}{\text{err}_{\text{test}}} % test error
\newcommand{\errexp}{\overline{\text{err}_{\text{test}}}} % avg training error

% lm
\newcommand{\thx}{\thetab^\top \xv} % linear model
\newcommand{\olsest}{(\Xmat^\top \Xmat)^{-1} \Xmat^\top \yv} % OLS estimator in LM 

% linear svm
\newcommand{\sv}{\operatorname{SV}} % supportvectors
\ifdefined\sl
\renewcommand{\sl}{\zeta} % slack variable
\else \newcommand{\sl}{\zeta} \fi
\newcommand{\slvec}{\left(\zeta^{(1)}, \zeta^{(n)}\right)} % slack variable vector
\newcommand{\sli}[1][i]{\zeta^{(#1)}} % i-th slack variable
\newcommand{\scptxi}{\scp{\thetav}{\xi}} % scalar prodct of theta and xi
\newcommand{\svmhplane}{\yi \left( \scp{\thetav}{\xi} + \theta_0 \right)} % SVM hyperplane (normalized)
\newcommand{\alphah}{\hat{\alpha}} % alpha-hat (basis fun coefficients)
\newcommand{\alphav}{\bm{\alpha}} % vector alpha (bold) (basis fun coefficients)
\newcommand{\alphavh}{\hat{\bm{\alpha}}} % vector alpha-hat (basis fun coefficients)
\newcommand{\dualobj}{\sumin \alpha_i - \frac{1}{2}\sumin \sumjn \alpha_i\alpha_j\yi \yi[j] \scp{\xi}{\xv^{(j)}}} % min objective in lin svm dual

% nonlinear svm
\newcommand{\HS}{\Phi} % H, hilbertspace
\newcommand{\phix}{\phi(\xv)} % feature map x
\newcommand{\phixt}{\phi(\tilde \xv)} % feature map x tilde
\newcommand{\kxxt}{k(\xv, \tilde \xv)} % kernel fun x, x tilde
\newcommand{\scptxifm}{\scp{\thetav}{\phi(\xi)}} % scalar prodct of theta and xi



\title{SL :\,: BASICS} % Package title in header, \, adds thin space between ::
\newcommand{\packagedescription}{ \invisible{x} % Package description in header
	% The \textbf{I2ML}: Introduction to Machine Learning course offers an introductory and applied overview of "supervised" Machine Learning. It is organized as a digital lecture.
}

\newlength{\columnheight} % Adjust depending on header height
\setlength{\columnheight}{84cm} 

\newtcolorbox{codebox}{%
	sharp corners,
	leftrule=0pt,
	rightrule=0pt,
	toprule=0pt,
	bottomrule=0pt,
	hbox}

\newtcolorbox{codeboxmultiline}[1][]{%
	sharp corners,
	leftrule=0pt,
	rightrule=0pt,
	toprule=0pt,
	bottomrule=0pt,
	#1}
	

	
\begin{document}
	\begin{frame}[fragile]{}
	\vspace{-8ex}
		\begin{columns}
			\begin{column}{.31\textwidth}
				\begin{beamercolorbox}[center]{postercolumn}
					\begin{minipage}{.98\textwidth}
						\parbox[t][\columnheight]{\textwidth}{
							%%%%%%%%%%%%%%%%%%%%%%%%%%%%%%%%%%%%%%%%%%%%%%%%%%%%%%%%%%%%%%%%%%%%%%%%%%%%%%%%
							% First Column begin
							%-------------------------------------------------------------------------------
							% Data
							%-------------------------------------------------------------------------------
							\begin{myblock}{Information Theory (Continuous)}
							%	
							%	
								Differential entropy: Continuous random variable $X$ with density function $f(x)$ and support $\Xspace:$
							%	
									$$ h(X) := h(f) := - \int_{\Xspace} f(x) \log(f(x)) dx $$
							%
								Joint differential entropy: Continuous random vector $X$ with density function $f(x)$ and support $\Xspace:$
							%	
									$$ h(X) = h(X_1, \ldots, X_n) = h(f) = - \int_{\Xspace} f(x) \log(f(x)) dx $$
							%
								Properties:
							%	
								\begin{enumerate}
									\setlength{\itemindent}{+.3in}
								\item $h(f)$ can be negative.
								\item $h(f)$ is additive for independent RVs.
								\item $h(f)$ is maximized by the multivariate normal, if we restrict 
								to all distributions with the same (co)variance, so
								$h(X) \leq \frac{1}{2} \ln(2 \pi e)^n |\Sigma|.$
								% We postpone the proof to a later chapter, as it is based on Kullback-Leibler divergence.
								% $H(X) \leq -g\frac{1}{g} \log_2(\frac{1}{g}) = log_2(g)$.
								\item Translation-invariant, $ h(X+a) = h(X)$. 
							%	\item $h(aX) = h(X) + \log |a|$.
								\item $h(AX) = h(X) + \log |A|$ for random vectors and matrix A.
								\end{enumerate}
							%
								\textbf{Conditional entropy} of $Y$ given $X$ (both continuous):
								%
								$$h(Y|X) = - \int f(x,y) \log f(x|y) dx dy.$$
								%
								\textbf{Mutual information:}
							%	 
								\begin{equation*}\begin{aligned}
										I(X ; Y) &= \int f(x,y) \log \frac{f(x,y)}{f(x)f(y)} dx dy.
									\end{aligned}
								\end{equation*}
							%
								\textbf{Cross-entropy} of two densities $p$ and $q$ on the same domain $\Xspace:$
								%
								$$ H_p(q) = \int q(x) \ln\left(\frac{1}{p(x)}\right) dx = - \int q(x) \ln\left(p(x)\right) dx $$
								%
								\textbf{Kullback-Leibler Divergence}:
								%Kullback-Leibler Divergence
								$$ D_{KL}(p \| q) = \E_p \left[\log \frac{p(X)}{q(X)}\right] = \int_{x \in \Xspace} p(x) \cdot \log \frac{p(x)}{q(x)} $$
								% 	
							\end{myblock}
							%
							%-------------------------------------------------------------------------------
							%   
							%-------------------------------------------------------------------------------
							\begin{myblock}{Hypothesis Space}
							%
								\textbf{Underfitting:} Failure to obtain a sufficiently low training error. \\
								\textbf{Overfitting:} Large difference in training and test error.\\
								
								\textbf{VC dimension:} General measure of the complexity of a function space.
								%
								The \textbf{VC dimension} of a class of binary-valued functions $\Hspace = \{h: \Xspace \to \{0, 1\}\}$ is defined to be the largest number of points in $\Xspace$ (in some configuration) that can be shattered by members of $\Hspace$. \\
								%
								Notation: $VC_p(\Hspace)$, where $p$ denotes the dimension of $\Xspace$.\\
								% 
								\textbf{Shattering:} A set of points is said to be \textbf{shattered} by a class of functions if  a member of this class can perfectly separate them no matter how we assign binary labels to the points.
								%
							\end{myblock}\vfill
							% End First Column
							%%%%%%%%%%%%%%%%%%%%%%%%%%%%%%%%%%%%%%%%%%%%%%%%%%%%%%%%%%%%%%%%%%%%%%%%%%%%%%%%
						}
					\end{minipage}
				\end{beamercolorbox}
			\end{column}
			\begin{column}{.31\textwidth}
				\begin{beamercolorbox}[center]{postercolumn}
					\begin{minipage}{.98\textwidth}
						\parbox[t][\columnheight]{\textwidth}{
							%%%%%%%%%%%%%%%%%%%%%%%%%%%%%%%%%%%%%%%%%%%%%%%%%%%%%%%%%%%%%%%%%%%%%%%%%%%%%%%%
							% Begin Second Column
							\begin{myblock}{Regularization}  
							%
								Regularized Empirical Risk:
								$$
								\riskrf = \riskef + \lambda \cdot J(f)  
								$$
								\begin{itemize}
									\setlength{\itemindent}{+.3in}
								%	
									\item $J(f)$ is the \textbf{complexity/roughness penalty} or \textbf{regularizer}.
									\item $\lambda > 0$ is the \textbf{complexity control} parameter. 
									\item For parameterized hypotheses: $\riskrt = \risket + \lambda \cdot J(\thetab)$. 
								\end{itemize}
								%
								Tackles the trade-off: \emph{maximizing} the fit (minimizing the train loss) vs.\ \emph{minimizing} the complexity of the model. \\
								
								%
								Regularization in the linear model ($\fx = \thetab^\top \xv$):
								%
								\begin{itemize}
									\setlength{\itemindent}{+.3in}
									%	
									\item Ridge regression: $J(\thetab) =  \|\thetab\|_2^2 = \thetab^\top \thetab.$
									\item Lasso regression: $J(\thetab) =  \|\thetab\|_1 = \sum_{j=1}^p |\theta_j|.$
									\item Elastic net regression: $J(\thetab) =  (\|\thetab\|_2^2,  \|\thetab\|_1)^\top$ and $\lambda=(\lambda_1,\lambda_2).$
									\item L0 regression: $J(\thetab) = \|\thetab\|_0 = \sum_{j=1}^p |\theta_j|^0.$
								%	
								\end{itemize}
								%
								\textbf{Early stopping:}
								%
								\begin{enumerate}
									\setlength{\itemindent}{+.3in}
									\item Split training data $\Dtrain$ into $\mathcal{D}_{\text{subtrain}}$ and $\mathcal{D}_{\text{val}}.$ 
									\item Train on $\mathcal{D}_{\text{subtrain}}$ and evaluate model using the validation set $\mathcal{D}_{\text{val}}$.
									\item Stop training when validation error stops decreasing.
									\item Use parameters of the previous step for the actual model.
								\end{enumerate}
							%
							\end{myblock} 
							%
							%
							\begin{myblock}{Linear Support Vector Machines}
								%	
								Signed distance to the separating hyperplane:
								$$
								d \left(f, \xi \right) = \frac{\yi \fxi}{\|\thetab\|} = \yi \frac{\thetab^T \xi + \theta_0}{\|\thetab\|}
								$$ 
								Distance of $f$ to the whole dataset $\D:$ 
								$
								\gamma = \min\limits_i \Big\{ d \left(f, \xi \right) \Big\}
								$
								
								\textbf{Primal linear hard-margin SVM:}
								%
								\begin{eqnarray*}
									& \min\limits_{\thetab, \theta_0} \quad & \frac{1}{2} \|\thetab\|^2 \\
									& \text{s.t.} & \,\,\yi  \left( \scp{\thetab}{\xi} + \theta_0 \right) \geq 1 \quad \forall\, i \in \nset
								\end{eqnarray*}
								%
								Support vectors: All instances $(\xi, \yi)$ with minimal margin
								$\yi  \fxi = 1$, fulfilling the inequality constraints with equality. 
								All have distance of $\gamma = 1 / \|\thetab\|$ from the separating hyperplane.
								
								\textbf{Dual linear hard-margin SVM:}
								%
								\begin{eqnarray*}
									& \max\limits_{\alphav \in \R^n} & \sum\nolimits_{i=1}^n \alpha_i - \frac{1}{2}\sum\nolimits_{i=1}^n\sum\nolimits_{j=1}^n\alpha_i\alpha_j\yi y^{(j)} \scp{\xi}{\xv^{(j)}} \\
									& \text{s.t.} & \sum\nolimits_{i=1}^n \alpha_i \yi = 0, 
									\quad \text{and} \quad \alpha_i \ge 0~\forall i \in \nset
								\end{eqnarray*}
								%
								Solution (if existing):
								%
								$$
								\thetah = \sum\nolimits_{i=1}^n \hat \alpha_i \yi \xi, \quad \theta_0 = \yi - \scp{\thetab}{\xi}.
								$$
								%
							\end{myblock}
							% End Second Column					
							%%%%%%%%%%%%%%%%%%%%%%%%%%%%%%%%%%%%%%%%%%%%%%%%%%%%%%%%%%%%%%%%%%%%%%%%%%%%%%%%
						}
					\end{minipage}
				\end{beamercolorbox}
			\end{column}
			\begin{column}{.31\textwidth}
				\begin{beamercolorbox}[center]{postercolumn}
					\begin{minipage}{.98\textwidth}
						\parbox[t][\columnheight]{\textwidth}{
						%%%%%%%%%%%%%%%%%%%%%%%%%%%%%%%%%%%%%%%%%%%%%%%%%%%%%%%%%%%%%%%%%%%%%%%%%%%%%%%%
						% Begin Third Column#
						%-------------------------------------------------------------------------------
						% Regression Losses 
						%------------------------------------------------------------------------------- 
						\begin{myblock}{} \vspace{-4ex}
						%	
							\textbf{Primal linear soft-margin SVM:} 	
							\begin{eqnarray*}
								& \min\limits_{\thetab, \thetab_0,\sli} & \frac{1}{2} \|\thetab\|^2 + C   \sum_{i=1}^n \sli \\
								& \text{s.t.} & \,\, \yi  \left( \scp{\thetab}{\xi} + \thetab_0 \right) \geq 1 - \sli \quad \forall\, i \in \nset,\\
								& \text{and} & \,\, \sli \geq 0 \quad \forall\, i \in \nset,\\
							\end{eqnarray*}
						%
							where the constant $C > 0$ controls trade-off between the two conflicting
							objectives of maximizing the size of the margin and minimizing the
							frequency and size of margin violations\\
							
						%
							\textbf{Dual linear soft-margin SVM:} 	
						%	
							\begin{eqnarray*}
								& \max\limits_{\alphav \in \R^n} & \sum_{i=1}^n \alpha_i - \frac{1}{2}\sum_{i=1}^n\sum_{j=1}^n\alpha_i\alpha_j\yi y^{(j)} \scp{\xi}{\xv^{(j)}} \\
								& \text{s.t. } & 0 \le \alpha_i \le C, \forall\, i \in \nset \quad \text{and} \quad  \sum_{i=1}^n \alpha_i \yi = 0
							\end{eqnarray*}
						%
						Support Vectors: All instances $(\xi, \yi)$ which 
						% 
						\begin{itemize}
							\setlength{\itemindent}{+.3in}
							\item are located exactly on the
							margin and have $y\fx=1$.
							\item are margin violators, with $y\fx < 1$, and have an associated positive slack $\sli > 0$. 
							They are misclassified if $\sli \geq 1$.
						\end{itemize}
						%
						Regularized empirical risk minimization representation:
						%
						$$ \risket = \frac{1}{2} \|\thetab\|^2 + C \sumin \Lxyi ;\; \Lyf = \max(1-yf, 0)$$
						%

						\end{myblock}

						\begin{myblock}{Kernels}
						%	
							\textbf{Mercer Kernel:} Continuous function
							$ k : \Xspace \times \Xspace \to \R $ fulfilling
						%	
							\begin{itemize}
								\setlength{\itemindent}{+.3in}
							\item Symmetry: $k(\xv, \tilde \xv) = k(\tilde \xv, \xv)$ for all
							$\xv, \tilde \xv \in \Xspace$.
							\item Positive definiteness: For each finite subset $\left\{\xv^{(1)}, \dots, \xv^{(n)}\right\}$
							the \textbf{kernel Gram matrix} $\bm{K} \in \R^{n \times n}$ with entries
							$K_{ij} = k(\xi, \xv^{(j)})$ is positive semi-definite.
							\end{itemize}
						%	
							Properties: For two Mercer kernels $k_1$ and~$k_2$:
						%	
							\begin{itemize}
								\setlength{\itemindent}{+.3in}
								\item For $\lambda \geq 0$, $\lambda \cdot k_1$ is a kernel.
								\item $k_1 + k_2$ is a kernel.
								\item $k_1 \cdot k_2$ is a kernel (thus also $k_1^n$).
							\end{itemize}
							Examples:
							\begin{itemize}
								\setlength{\itemindent}{+.3in}
								\item Linear kernel: $k(\xv, \tilde \xv) = \xv^\top \tilde \xv$
								\item Homogeneous polynomial kernel:
								$ k(\xv, \xtil) = (\xv^T \xtil)^d, \text{ for } d \in \N$
								\item Nonhomogeneous polynomial k	ernel: $k(\xv, \tilde \xv) = (\xv^\top \tilde \xv + b)^d, \text{ for } b\geq 0, d \in \N$
								\item Radial Gaussian kernel (RBF):
								$k(\xv, \tilde \xv) = \exp(-\frac{\|\xv - \tilde \xv\|^2}{2\sigma^2})$ 
								or 
								$k(\xv, \tilde \xv) = \exp(-\gamma \|\xv - \tilde \xv\|^2), ~ \gamma>0$
							\end{itemize}
						\end{myblock} 
						%
						%
						%-------------------------------------------------------------------------------
						% Classification 
						%------------------------------------------------------------------------------- 

						%\begin{myblock}{Classification}
						% 				    We want to assign new observations to known categories according to criteria learned from a training set.  
						%             \vspace*{1ex}
						%             

						%$y \in \Yspace = \gset : $ categorical output variable (label)\\ 

						%\textbf{Classification} usually means to construct $g$ \textbf{discriminant functions}:
						
						%$f_1(\xv), \ldots, \fgx$, so that we choose our class as \\ $h(\xv) = \argmax_{k \in \gset} \fkx$ \\

						%\textbf{Linear Classifier:} functions $\fkx$ can be specified as linear functions\\

						% \hspace*{1ex}\textbf{Note: }All linear classifiers can represent non-linear decision boundaries \hspace*{1ex}in our original input space if we include derived features. For example: \hspace*{1ex}higher order interactions, polynomials or other transformations of x in \hspace*{1ex}the model.

						%\textbf{Binary classification: }If only 2 classes ($\Yspace = \setzo$ or  $\Yspace = \setmp$) exist, we can use a single discriminant function $\fx = f_{1}(\xv) - f_{2}(\xv)$.  \\


						% \textbf{Generative approach }models $\pdfxyk$, usually by making some assumptions about the structure of these distributions and employs the Bayes theorem: 
						% $\pikx = \postk \propto \pdfxyk \pik$. \\ %It allows the computation of \hspace*{1ex}$\pikx$. \\
						% \textbf{Examples}: Linear discriminant analysis (LDA), Quadratic discriminant analysis (QDA), Naive Bayes\\
						% 
						% \textbf{Discriminant approach }tries to optimize the discriminant functions directly, usually via empirical risk minimization:\\ 
						% $ \fh = \argmin_{f \in \Hspace} \riske(f) = \argmin_{f \in \Hspace} \sumin \Lxyi.$\\
						% \textbf{Examples}: Logistic/softmax regression, kNN


						%\end{myblock}

						%-------------------------------------------------------------------------------
						% HRO - Components of Learning 
						%-------------------------------------------------------------------------------          
						%\begin{myblock}{Components of Learning}

						%\textbf{Learning = Hypothesis space + Risk + Optimization} \\
						%\phantom{\textbf{Learning}} \textbf{= }$ \Hspace + \risket + \argmin_{\thetab \in \Theta} 
						%\risket$

						% 
						% \textbf{Learning &= Hypothesis space &+ Risk  &+ Optimization} \\
						% &= $\Hspace &+ \risket &+ \argmin_{\thetab \in \Theta} \risket$
						% 
						% \textbf{Hypothesis space: } Defines (and restricts!) what kind of model $f$
						% can be learned from the data.
						% 
						% Examples: linear functions, decision trees
						% 
						% \vspace*{0.5ex}
						% 
						% \textbf{Risk: } Quantifies how well a model performs on a given
						% data set. This allows us to rank candidate models in order to choose the best one.
						% 
						% Examples: squared error, negative (log-)likelihood
						% 
						% \vspace*{0.5ex}
						% 
						% \textbf{Optimization: } Defines how to search for the best model, i.e., the model with the smallest {risk}, in the hypothesis space.
						% 
						% Examples: gradient descent, quadratic programming


						%\end{myblock}
						% End Third Column
						%%%%%%%%%%%%%%%%%%%%%%%%%%%%%%%%%%%%%%%%%%%%%%%%%%%%%%%%%%%%%%%%%%%%%%%%%%%%%%%%%%%%%%%%%%%%%%%
					}
					\end{minipage}
				\end{beamercolorbox}
			\end{column}
		\end{columns}
	\end{frame}
\end{document}
