\documentclass{beamer}


\usepackage[orientation=landscape,size=a0,scale=1.4,debug]{beamerposter}
\mode<presentation>{\usetheme{mlr}}


\usepackage[utf8]{inputenc} % UTF-8
\usepackage[english]{babel} % Language
\usepackage{hyperref} % Hyperlinks
\usepackage{ragged2e} % Text position
\usepackage[export]{adjustbox} % Image position
\usepackage[most]{tcolorbox}
\usepackage{amsmath}
\usepackage{mathtools}
\usepackage{dsfont}
\usepackage{verbatim}
\usepackage{amsmath}
\usepackage{amsfonts}
\usepackage{csquotes}
\usepackage{multirow}
\usepackage{longtable}
\usepackage{enumerate}
\usepackage[absolute,overlay]{textpos}
\usepackage{psfrag}
\usepackage{algorithm}
\usepackage{algpseudocode}
\usepackage{eqnarray}
\usepackage{arydshln}
\usepackage{tabularx}
\usepackage{placeins}
\usepackage{tikz}
\usepackage{setspace}
\usepackage{colortbl}
\usepackage{mathtools}
\usepackage{wrapfig}
\usepackage{bm}
\usepackage{nicefrac}

% math spaces
\ifdefined\N                                                                
\renewcommand{\N}{\mathds{N}} % N, naturals
\else \newcommand{\N}{\mathds{N}} \fi 
\newcommand{\Z}{\mathds{Z}} % Z, integers
\newcommand{\Q}{\mathds{Q}} % Q, rationals
\newcommand{\R}{\mathds{R}} % R, reals
\ifdefined\C 
  \renewcommand{\C}{\mathds{C}} % C, complex
\else \newcommand{\C}{\mathds{C}} \fi
\newcommand{\continuous}{\mathcal{C}} % C, space of continuous functions
\newcommand{\M}{\mathcal{M}} % machine numbers
\newcommand{\epsm}{\epsilon_m} % maximum error

% counting / finite sets
\newcommand{\setzo}{\{0, 1\}} % set 0, 1
\newcommand{\setmp}{\{-1, +1\}} % set -1, 1
\newcommand{\unitint}{[0, 1]} % unit interval

% basic math stuff
\newcommand{\xt}{\tilde x} % x tilde
\newcommand{\argmax}{\operatorname{arg\,max}} % argmax
\newcommand{\argmin}{\operatorname{arg\,min}} % argmin
\newcommand{\argminlim}{\mathop{\mathrm{arg\,min}}\limits} % argmax with limits
\newcommand{\argmaxlim}{\mathop{\mathrm{arg\,max}}\limits} % argmin with limits  
\newcommand{\sign}{\operatorname{sign}} % sign, signum
\newcommand{\I}{\mathbb{I}} % I, indicator
\newcommand{\order}{\mathcal{O}} % O, order
\newcommand{\pd}[2]{\frac{\partial{#1}}{\partial #2}} % partial derivative
\newcommand{\floorlr}[1]{\left\lfloor #1 \right\rfloor} % floor
\newcommand{\ceillr}[1]{\left\lceil #1 \right\rceil} % ceiling

% sums and products
\newcommand{\sumin}{\sum\limits_{i=1}^n} % summation from i=1 to n
\newcommand{\sumim}{\sum\limits_{i=1}^m} % summation from i=1 to m
\newcommand{\sumjn}{\sum\limits_{j=1}^n} % summation from j=1 to p
\newcommand{\sumjp}{\sum\limits_{j=1}^p} % summation from j=1 to p
\newcommand{\sumik}{\sum\limits_{i=1}^k} % summation from i=1 to k
\newcommand{\sumkg}{\sum\limits_{k=1}^g} % summation from k=1 to g
\newcommand{\sumjg}{\sum\limits_{j=1}^g} % summation from j=1 to g
\newcommand{\meanin}{\frac{1}{n} \sum\limits_{i=1}^n} % mean from i=1 to n
\newcommand{\meanim}{\frac{1}{m} \sum\limits_{i=1}^m} % mean from i=1 to n
\newcommand{\meankg}{\frac{1}{g} \sum\limits_{k=1}^g} % mean from k=1 to g
\newcommand{\prodin}{\prod\limits_{i=1}^n} % product from i=1 to n
\newcommand{\prodkg}{\prod\limits_{k=1}^g} % product from k=1 to g
\newcommand{\prodjp}{\prod\limits_{j=1}^p} % product from j=1 to p

% linear algebra
\newcommand{\one}{\boldsymbol{1}} % 1, unitvector
\newcommand{\zero}{\mathbf{0}} % 0-vector
\newcommand{\id}{\boldsymbol{I}} % I, identity
\newcommand{\diag}{\operatorname{diag}} % diag, diagonal
\newcommand{\trace}{\operatorname{tr}} % tr, trace
\newcommand{\spn}{\operatorname{span}} % span
\newcommand{\scp}[2]{\left\langle #1, #2 \right\rangle} % <.,.>, scalarproduct
\newcommand{\mat}[1]{\begin{pmatrix} #1 \end{pmatrix}} % short pmatrix command
\newcommand{\Amat}{\mathbf{A}} % matrix A
\newcommand{\Deltab}{\mathbf{\Delta}} % error term for vectors

% basic probability + stats
\renewcommand{\P}{\mathds{P}} % P, probability
\newcommand{\E}{\mathds{E}} % E, expectation
\newcommand{\var}{\mathsf{Var}} % Var, variance
\newcommand{\cov}{\mathsf{Cov}} % Cov, covariance
\newcommand{\corr}{\mathsf{Corr}} % Corr, correlation
\newcommand{\normal}{\mathcal{N}} % N of the normal distribution
\newcommand{\iid}{\overset{i.i.d}{\sim}} % dist with i.i.d superscript
\newcommand{\distas}[1]{\overset{#1}{\sim}} % ... is distributed as ...

% machine learning
\newcommand{\Xspace}{\mathcal{X}} % X, input space
\newcommand{\Yspace}{\mathcal{Y}} % Y, output space
\newcommand{\nset}{\{1, \ldots, n\}} % set from 1 to n
\newcommand{\pset}{\{1, \ldots, p\}} % set from 1 to p
\newcommand{\gset}{\{1, \ldots, g\}} % set from 1 to g
\newcommand{\Pxy}{\mathbb{P}_{xy}} % P_xy
\newcommand{\Exy}{\mathbb{E}_{xy}} % E_xy: Expectation over random variables xy
\newcommand{\xv}{\mathbf{x}} % vector x (bold)
\newcommand{\xtil}{\tilde{\mathbf{x}}} % vector x-tilde (bold)
\newcommand{\yv}{\mathbf{y}} % vector y (bold)
\newcommand{\xy}{(\xv, y)} % observation (x, y)
\newcommand{\xvec}{\left(x_1, \ldots, x_p\right)^\top} % (x1, ..., xp) 
\newcommand{\Xmat}{\mathbf{X}} % Design matrix
\newcommand{\allDatasets}{\mathds{D}} % The set of all datasets
\newcommand{\allDatasetsn}{\mathds{D}_n}  % The set of all datasets of size n 
\newcommand{\D}{\mathcal{D}} % D, data
\newcommand{\Dn}{\D_n} % D_n, data of size n
\newcommand{\Dtrain}{\mathcal{D}_{\text{train}}} % D_train, training set
\newcommand{\Dtest}{\mathcal{D}_{\text{test}}} % D_test, test set
\newcommand{\xyi}[1][i]{\left(\xv^{(#1)}, y^{(#1)}\right)} % (x^i, y^i), i-th observation
\newcommand{\Dset}{\left( \xyi[1], \ldots, \xyi[n]\right)} % {(x1,y1)), ..., (xn,yn)}, data
\newcommand{\defAllDatasetsn}{(\Xspace \times \Yspace)^n} % Def. of the set of all datasets of size n 
\newcommand{\defAllDatasets}{\bigcup_{n \in \N}(\Xspace \times \Yspace)^n} % Def. of the set of all datasets 
\newcommand{\xdat}{\left\{ \xv^{(1)}, \ldots, \xv^{(n)}\right\}} % {x1, ..., xn}, input data
\newcommand{\ydat}{\left\{ \yv^{(1)}, \ldots, \yv^{(n)}\right\}} % {y1, ..., yn}, input data
\newcommand{\yvec}{\left(y^{(1)}, \hdots, y^{(n)}\right)^\top} % (y1, ..., yn), vector of outcomes
\renewcommand{\xi}[1][i]{\xv^{(#1)}} % x^i, i-th observed value of x
\newcommand{\yi}[1][i]{y^{(#1)}} % y^i, i-th observed value of y 
\newcommand{\xivec}{\left(x^{(i)}_1, \ldots, x^{(i)}_p\right)^\top} % (x1^i, ..., xp^i), i-th observation vector
\newcommand{\xj}{\xv_j} % x_j, j-th feature
\newcommand{\xjvec}{\left(x^{(1)}_j, \ldots, x^{(n)}_j\right)^\top} % (x^1_j, ..., x^n_j), j-th feature vector
\newcommand{\phiv}{\mathbf{\phi}} % Basis transformation function phi
\newcommand{\phixi}{\mathbf{\phi}^{(i)}} % Basis transformation of xi: phi^i := phi(xi)

%%%%%% ml - models general
\newcommand{\lamv}{\bm{\lambda}} % lambda vector, hyperconfiguration vector
\newcommand{\Lam}{\bm{\Lambda}}	 % Lambda, space of all hpos
% Inducer / Inducing algorithm
\newcommand{\preimageInducer}{\left(\defAllDatasets\right)\times\Lam} % Set of all datasets times the hyperparameter space
\newcommand{\preimageInducerShort}{\allDatasets\times\Lam} % Set of all datasets times the hyperparameter space
% Inducer / Inducing algorithm
\newcommand{\ind}{\mathcal{I}} % Inducer, inducing algorithm, learning algorithm 

% continuous prediction function f
\newcommand{\ftrue}{f_{\text{true}}}  % True underlying function (if a statistical model is assumed)
\newcommand{\ftruex}{\ftrue(\xv)} % True underlying function (if a statistical model is assumed)
\newcommand{\fx}{f(\xv)} % f(x), continuous prediction function
\newcommand{\fdomains}{f: \Xspace \rightarrow \R^g} % f with domain and co-domain
\newcommand{\Hspace}{\mathcal{H}} % hypothesis space where f is from
\newcommand{\fbayes}{f^{\ast}} % Bayes-optimal model
\newcommand{\fxbayes}{f^{\ast}(\xv)} % Bayes-optimal model
\newcommand{\fkx}[1][k]{f_{#1}(\xv)} % f_j(x), discriminant component function
\newcommand{\fh}{\hat{f}} % f hat, estimated prediction function
\newcommand{\fxh}{\fh(\xv)} % fhat(x)
\newcommand{\fxt}{f(\xv ~|~ \thetab)} % f(x | theta)
\newcommand{\fxi}{f\left(\xv^{(i)}\right)} % f(x^(i))
\newcommand{\fxih}{\hat{f}\left(\xv^{(i)}\right)} % f(x^(i))
\newcommand{\fxit}{f\left(\xv^{(i)} ~|~ \thetab\right)} % f(x^(i) | theta)
\newcommand{\fhD}{\fh_{\D}} % fhat_D, estimate of f based on D
\newcommand{\fhDtrain}{\fh_{\Dtrain}} % fhat_Dtrain, estimate of f based on D
\newcommand{\fhDnlam}{\fh_{\Dn, \lamv}} %model learned on Dn with hp lambda
\newcommand{\fhDlam}{\fh_{\D, \lamv}} %model learned on D with hp lambda
\newcommand{\fhDnlams}{\fh_{\Dn, \lamv^\ast}} %model learned on Dn with optimal hp lambda 
\newcommand{\fhDlams}{\fh_{\D, \lamv^\ast}} %model learned on D with optimal hp lambda 

% discrete prediction function h
\newcommand{\hx}{h(\xv)} % h(x), discrete prediction function
\newcommand{\hh}{\hat{h}} % h hat
\newcommand{\hxh}{\hat{h}(\xv)} % hhat(x)
\newcommand{\hxt}{h(\xv | \thetab)} % h(x | theta)
\newcommand{\hxi}{h\left(\xi\right)} % h(x^(i))
\newcommand{\hxit}{h\left(\xi ~|~ \thetab\right)} % h(x^(i) | theta)
\newcommand{\hbayes}{h^{\ast}} % Bayes-optimal classification model
\newcommand{\hxbayes}{h^{\ast}(\xv)} % Bayes-optimal classification model

% yhat
\newcommand{\yh}{\hat{y}} % yhat for prediction of target
\newcommand{\yih}{\hat{y}^{(i)}} % yhat^(i) for prediction of ith targiet
\newcommand{\resi}{\yi- \yih}

% theta
\newcommand{\thetah}{\hat{\theta}} % theta hat
\newcommand{\thetab}{\bm{\theta}} % theta vector
\newcommand{\thetabh}{\bm{\hat\theta}} % theta vector hat
\newcommand{\thetat}[1][t]{\thetab^{[#1]}} % theta^[t] in optimization
\newcommand{\thetatn}[1][t]{\thetab^{[#1 +1]}} % theta^[t+1] in optimization
\newcommand{\thetahDnlam}{\thetabh_{\Dn, \lamv}} %theta learned on Dn with hp lambda
\newcommand{\thetahDlam}{\thetabh_{\D, \lamv}} %theta learned on D with hp lambda
\newcommand{\mint}{\min_{\thetab \in \Theta}} % min problem theta
\newcommand{\argmint}{\argmin_{\thetab \in \Theta}} % argmin theta

% densities + probabilities
% pdf of x 
\newcommand{\pdf}{p} % p
\newcommand{\pdfx}{p(\xv)} % p(x)
\newcommand{\pixt}{\pi(\xv~|~ \thetab)} % pi(x|theta), pdf of x given theta
\newcommand{\pixit}[1][i]{\pi\left(\xi[#1] ~|~ \thetab\right)} % pi(x^i|theta), pdf of x given theta
\newcommand{\pixii}{\pi\left(\xi\right)} % pi(x^i), pdf of i-th x 

% pdf of (x, y)
\newcommand{\pdfxy}{p(\xv,y)} % p(x, y)
\newcommand{\pdfxyt}{p(\xv, y ~|~ \thetab)} % p(x, y | theta)
\newcommand{\pdfxyit}{p\left(\xi, \yi ~|~ \thetab\right)} % p(x^(i), y^(i) | theta)

% pdf of x given y
\newcommand{\pdfxyk}[1][k]{p(\xv | y= #1)} % p(x | y = k)
\newcommand{\lpdfxyk}[1][k]{\log p(\xv | y= #1)} % log p(x | y = k)
\newcommand{\pdfxiyk}[1][k]{p\left(\xi | y= #1 \right)} % p(x^i | y = k)

% prior probabilities
\newcommand{\pik}[1][k]{\pi_{#1}} % pi_k, prior
\newcommand{\lpik}[1][k]{\log \pi_{#1}} % log pi_k, log of the prior
\newcommand{\pit}{\pi(\thetab)} % Prior probability of parameter theta

% posterior probabilities
\newcommand{\post}{\P(y = 1 ~|~ \xv)} % P(y = 1 | x), post. prob for y=1
\newcommand{\postk}[1][k]{\P(y = #1 ~|~ \xv)} % P(y = k | y), post. prob for y=k
\newcommand{\pidomains}{\pi: \Xspace \rightarrow \unitint} % pi with domain and co-domain
\newcommand{\pibayes}{\pi^{\ast}} % Bayes-optimal classification model
\newcommand{\pixbayes}{\pi^{\ast}(\xv)} % Bayes-optimal classification model
\newcommand{\pix}{\pi(\xv)} % pi(x), P(y = 1 | x)
\newcommand{\piv}{\bm{\pi}} % pi, bold, as vector
\newcommand{\pikx}[1][k]{\pi_{#1}(\xv)} % pi_k(x), P(y = k | x)
\newcommand{\pikxt}[1][k]{\pi_{#1}(\xv ~|~ \thetab)} % pi_k(x | theta), P(y = k | x, theta)
\newcommand{\pixh}{\hat \pi(\xv)} % pi(x) hat, P(y = 1 | x) hat
\newcommand{\pikxh}[1][k]{\hat \pi_{#1}(\xv)} % pi_k(x) hat, P(y = k | x) hat
\newcommand{\pixih}{\hat \pi(\xi)} % pi(x^(i)) with hat
\newcommand{\pikxih}[1][k]{\hat \pi_{#1}(\xi)} % pi_k(x^(i)) with hat
\newcommand{\pdfygxt}{p(y ~|~\xv, \thetab)} % p(y | x, theta)
\newcommand{\pdfyigxit}{p\left(\yi ~|~\xi, \thetab\right)} % p(y^i |x^i, theta)
\newcommand{\lpdfygxt}{\log \pdfygxt } % log p(y | x, theta)
\newcommand{\lpdfyigxit}{\log \pdfyigxit} % log p(y^i |x^i, theta)

% probababilistic
\newcommand{\bayesrulek}[1][k]{\frac{\P(\xv | y= #1) \P(y= #1)}{\P(\xv)}} % Bayes rule
\newcommand{\muk}{\bm{\mu_k}} % mean vector of class-k Gaussian (discr analysis) 

% residual and margin
\newcommand{\eps}{\epsilon} % residual, stochastic
\newcommand{\epsi}{\epsilon^{(i)}} % epsilon^i, residual, stochastic
\newcommand{\epsh}{\hat{\epsilon}} % residual, estimated
\newcommand{\yf}{y \fx} % y f(x), margin
\newcommand{\yfi}{\yi \fxi} % y^i f(x^i), margin
\newcommand{\Sigmah}{\hat \Sigma} % estimated covariance matrix
\newcommand{\Sigmahj}{\hat \Sigma_j} % estimated covariance matrix for the j-th class

% ml - loss, risk, likelihood
\newcommand{\Lyf}{L\left(y, f\right)} % L(y, f), loss function
\newcommand{\Lypi}{L\left(y, \pi\right)} % L(y, pi), loss function
\newcommand{\Lxy}{L\left(y, \fx\right)} % L(y, f(x)), loss function
\newcommand{\Lxyi}{L\left(\yi, \fxi\right)} % loss of observation
\newcommand{\Lxyt}{L\left(y, \fxt\right)} % loss with f parameterized
\newcommand{\Lxyit}{L\left(\yi, \fxit\right)} % loss of observation with f parameterized
\newcommand{\Lxym}{L\left(\yi, f\left(\bm{\tilde{x}}^{(i)} ~|~ \thetab\right)\right)} % loss of observation with f parameterized
\newcommand{\Lpixy}{L\left(y, \pix\right)} % loss in classification
\newcommand{\Lpiv}{L\left(y, \piv\right)} % loss in classification
\newcommand{\Lpixyi}{L\left(\yi, \pixii\right)} % loss of observation in classification
\newcommand{\Lpixyt}{L\left(y, \pixt\right)} % loss with pi parameterized
\newcommand{\Lpixyit}{L\left(\yi, \pixit\right)} % loss of observation with pi parameterized
\newcommand{\Lhxy}{L\left(y, \hx\right)} % L(y, h(x)), loss function on discrete classes
\newcommand{\Lr}{L\left(r\right)} % L(r), loss defined on residual (reg) / margin (classif)
\newcommand{\lone}{|y - \fx|} % L1 loss
\newcommand{\ltwo}{\left(y - \fx\right)^2} % L2 loss
\newcommand{\lbernoullimp}{\ln(1 + \exp(-y \cdot \fx))} % Bernoulli loss for -1, +1 encoding
\newcommand{\lbernoullizo}{- y \cdot \fx + \log(1 + \exp(\fx))} % Bernoulli loss for 0, 1 encoding
\newcommand{\lcrossent}{- y \log \left(\pix\right) - (1 - y) \log \left(1 - \pix\right)} % cross-entropy loss
\newcommand{\lbrier}{\left(\pix - y \right)^2} % Brier score
\newcommand{\risk}{\mathcal{R}} % R, risk
\newcommand{\riskbayes}{\mathcal{R}^\ast}
\newcommand{\riskf}{\risk(f)} % R(f), risk
\newcommand{\riskdef}{\E_{y|\xv}\left(\Lxy \right)} % risk def (expected loss)
\newcommand{\riskt}{\mathcal{R}(\thetab)} % R(theta), risk
\newcommand{\riske}{\mathcal{R}_{\text{emp}}} % R_emp, empirical risk w/o factor 1 / n
\newcommand{\riskeb}{\bar{\mathcal{R}}_{\text{emp}}} % R_emp, empirical risk w/ factor 1 / n
\newcommand{\riskef}{\riske(f)} % R_emp(f)
\newcommand{\risket}{\mathcal{R}_{\text{emp}}(\thetab)} % R_emp(theta)
\newcommand{\riskr}{\mathcal{R}_{\text{reg}}} % R_reg, regularized risk
\newcommand{\riskrt}{\mathcal{R}_{\text{reg}}(\thetab)} % R_reg(theta)
\newcommand{\riskrf}{\riskr(f)} % R_reg(f)
\newcommand{\riskrth}{\hat{\mathcal{R}}_{\text{reg}}(\thetab)} % hat R_reg(theta)
\newcommand{\risketh}{\hat{\mathcal{R}}_{\text{emp}}(\thetab)} % hat R_emp(theta)
\newcommand{\LL}{\mathcal{L}} % L, likelihood
\newcommand{\LLt}{\mathcal{L}(\thetab)} % L(theta), likelihood
\newcommand{\LLtx}{\mathcal{L}(\thetab | \xv)} % L(theta|x), likelihood
\newcommand{\logl}{\ell} % l, log-likelihood
\newcommand{\loglt}{\logl(\thetab)} % l(theta), log-likelihood
\newcommand{\logltx}{\logl(\thetab | \xv)} % l(theta|x), log-likelihood
\newcommand{\errtrain}{\text{err}_{\text{train}}} % training error
\newcommand{\errtest}{\text{err}_{\text{test}}} % test error
\newcommand{\errexp}{\overline{\text{err}_{\text{test}}}} % avg training error

% lm
\newcommand{\thx}{\thetab^\top \xv} % linear model
\newcommand{\olsest}{(\Xmat^\top \Xmat)^{-1} \Xmat^\top \yv} % OLS estimator in LM 

% linear svm
\newcommand{\sv}{\operatorname{SV}} % supportvectors
\ifdefined\sl
\renewcommand{\sl}{\zeta} % slack variable
\else \newcommand{\sl}{\zeta} \fi
\newcommand{\slvec}{\left(\zeta^{(1)}, \zeta^{(n)}\right)} % slack variable vector
\newcommand{\sli}[1][i]{\zeta^{(#1)}} % i-th slack variable
\newcommand{\scptxi}{\scp{\thetav}{\xi}} % scalar prodct of theta and xi
\newcommand{\svmhplane}{\yi \left( \scp{\thetav}{\xi} + \theta_0 \right)} % SVM hyperplane (normalized)
\newcommand{\alphah}{\hat{\alpha}} % alpha-hat (basis fun coefficients)
\newcommand{\alphav}{\bm{\alpha}} % vector alpha (bold) (basis fun coefficients)
\newcommand{\alphavh}{\hat{\bm{\alpha}}} % vector alpha-hat (basis fun coefficients)
\newcommand{\dualobj}{\sumin \alpha_i - \frac{1}{2}\sumin \sumjn \alpha_i\alpha_j\yi \yi[j] \scp{\xi}{\xv^{(j)}}} % min objective in lin svm dual

% nonlinear svm
\newcommand{\HS}{\Phi} % H, hilbertspace
\newcommand{\phix}{\phi(\xv)} % feature map x
\newcommand{\phixt}{\phi(\tilde \xv)} % feature map x tilde
\newcommand{\kxxt}{k(\xv, \tilde \xv)} % kernel fun x, x tilde
\newcommand{\scptxifm}{\scp{\thetav}{\phi(\xi)}} % scalar prodct of theta and xi

% ml - Gaussian Process

\newcommand{\fvec}{\left[f\left(\xi[1]\right), \dots, f\left(\xi[n]\right) \right]} % function vector
\newcommand{\fv}{\mathbf{f}} % function vector
\newcommand{\kv}{\mathbf{k}} % cov matrix partition
\newcommand{\kxxp}{k\left(\xv, \xv^{\prime} \right)} % cov of x, x'
\newcommand{\kxij}[2]{k\left(\xi, \xi[j] \right)} % cov of x_i, x_j
\newcommand{\mv}{\mathbf{m}} % GP mean vector
\newcommand{\Kmat}{\mathbf{K}} % GP cov matrix
\newcommand{\gaussmk}{\normal(\mv, \Kmat)} % Gaussian w/ mean vec, cov mat
\newcommand{\gp}{\mathcal{GP}\left(m(\xv), \kxxp \right)} % Gaussian Process Definition
\newcommand{\ls}{\ell} % length-scale
\newcommand{\sqexpkernel}{\exp \left(- \frac{\| \xv - \xv^{\prime} \|^2}{2 \ls^2} \right)} % squared exponential kernel

% GP prediction
\newcommand{\fstarvec}{\left[f\left(\xi[1]_{\ast}\right), \dots, f\left(\xi[m]_{\ast}\right) \right]} % pred function vector
\newcommand{\kstar}{\kv_{\ast}} % cov of new obs with x
\newcommand{\kstarstar}{\kv_{\ast \ast}} % cov of new obs
\newcommand{\Kstar}{\Kmat_{\ast}} % cov mat of new obs with x
\newcommand{\Kstarstar}{\Kmat_{\ast \ast}} % cov mat of new obs
\newcommand{\preddistsingle}{f_{\ast} ~|~ \xv_{\ast}, \Xmat, \fv} % predictive distribution for single pred
\newcommand{\preddistdefsingle}{\normal(\kstar^\top\Kmat^{-1}\fv, \kstarstar - \kstar^\top \Kmat ^{-1}\kstar)} % Gaussian distribution for single pred
\newcommand{\preddist}{f_{\ast} ~|~ \Xmat_{\ast}, \Xmat, \fv} % predictive distribution
\newcommand{\preddistdef}{\normal(\Kstar^\top\Kmat^{-1}\fv, \Kstarstar - \Kstar^\top \Kmat ^{-1}\Kstar)} % Gaussian predictive distribution

% ml - bagging, random forest
\newcommand{\bl}[1][m]{b^{[#1]}} % baselearner, default m
\newcommand{\blh}[1][m]{\hat{b}^{[#1]}} % estimated base learner, default m 
\newcommand{\blx}[1][m]{b^{[#1]}(\xv)} % baselearner, default m
\newcommand{\fM}{f^{[M]}(\xv)} % ensembled predictor
\newcommand{\fMh}{\hat f^{[M]}(\xv)} % estimated ensembled predictor
\newcommand{\ambifM}{\Delta\left(\fM\right)} % ambiguity/instability of ensemble
\newcommand{\betam}[1][m]{\beta^{[#1]}} % weight of basemodel m
\newcommand{\betamh}[1][m]{\hat{\beta}^{[#1]}} % weight of basemodel m with hat
\newcommand{\betaM}{\beta^{[M]}} % last baselearner

% ml - boosting
\newcommand{\fm}[1][m]{f^{[#1]}} % prediction in iteration m
\newcommand{\fmh}[1][m]{\hat{f}^{[#1]}} % prediction in iteration m
\newcommand{\fmd}[1][m]{f^{[#1-1]}} % prediction m-1
\newcommand{\fmdh}[1][m]{\hat{f}^{[#1-1]}} % prediction m-1
\newcommand{\errm}[1][m]{\text{err}^{[#1]}} % weighted in-sample misclassification rate
\newcommand{\wm}[1][m]{w^{[#1]}} % weight vector of basemodel m
\newcommand{\wmi}[1][m]{w^{[#1](i)}} % weight of obs i of basemodel m
\newcommand{\thetam}[1][m]{\thetab^{[#1]}} % parameters of basemodel m
\newcommand{\thetamh}[1][m]{\hat{\thetab}^{[#1]}} % parameters of basemodel m with hat
\newcommand{\blxt}[1][m]{b(\xv, \thetab^{[#1]})} % baselearner, default m
\newcommand{\ens}{\sum_{m=1}^M \betam \blxt} % ensemble
\newcommand{\rmm}[1][m]{\tilde{r}^{[#1]}} % pseudo residuals
\newcommand{\rmi}[1][m]{\tilde{r}^{[#1](i)}} % pseudo residuals
\newcommand{\Rtm}[1][m]{R_{t}^{[#1]}} % terminal-region
\newcommand{\Tm}[1][m]{T^{[#1]}} % terminal-region
\newcommand{\ctm}[1][m]{c_t^{[#1]}} % mean, terminal-regions
\newcommand{\ctmh}[1][m]{\hat{c}_t^{[#1]}} % mean, terminal-regions with hat
\newcommand{\ctmt}[1][m]{\tilde{c}_t^{[#1]}} % mean, terminal-regions
\newcommand{\Lp}{L^\prime}
\newcommand{\Ldp}{L^{\prime\prime}}
\newcommand{\Lpleft}{\Lp_{\text{left}}}
% ml - trees, extra trees

\newcommand{\Np}{\mathcal{N}} % (Parent) node N
\newcommand{\Npk}{\Np_k} % node N_k
\newcommand{\Nl}{\Np_1}	% Left node N_1
\newcommand{\Nr}{\Np_2} % Right node N_2
\newcommand{\pikN}[1][k]{\pi_#1^{(\Np)}} % class probability node N
\newcommand{\pikNh}[1][k]{\hat\pi_#1^{(\Np)}} % estimated class probability node N
\newcommand{\pikNlh}[1][k]{\hat\pi_#1^{(\Nl)}} % estimated class probability left node
\newcommand{\pikNrh}[1][k]{\hat\pi_#1^{(\Nr)}} % estimated class probability right node
	


\title{SL :\,: BASICS} % Package title in header, \, adds thin space between ::
\newcommand{\packagedescription}{ \invisible{x} % Package description in header
	% The \textbf{I2ML}: Introduction to Machine Learning course offers an introductory and applied overview of "supervised" Machine Learning. It is organized as a digital lecture.
}

\newlength{\columnheight} % Adjust depending on header height
\setlength{\columnheight}{84cm} 

\newtcolorbox{codebox}{%
	sharp corners,
	leftrule=0pt,
	rightrule=0pt,
	toprule=0pt,
	bottomrule=0pt,
	hbox}

\newtcolorbox{codeboxmultiline}[1][]{%
	sharp corners,
	leftrule=0pt,
	rightrule=0pt,
	toprule=0pt,
	bottomrule=0pt,
	#1}
	

	
\begin{document}
	\begin{frame}[fragile]{}
		\vspace{-8ex}
		\begin{columns}
			\begin{column}{.31\textwidth}
				\begin{beamercolorbox}[center]{postercolumn}
					\begin{minipage}{.98\textwidth}
						\parbox[t][\columnheight]{\textwidth}{
							%%%%%%%%%%%%%%%%%%%%%%%%%%%%%%%%%%%%%%%%%%%%%%%%%%%%%%%%%%%%%%%%%%%%%%%%%%%%%%%%
							% First Column begin
							%-------------------------------------------------------------------------------
							% Data
							%-------------------------------------------------------------------------------
							\begin{myblock}{Nonlinear Support Vector Machines}
							%	
							%	
								
								\textbf{Dual kernelized soft-margin SVM:}
							%	
									\begin{eqnarray*}
										& \max_{\alpha} & \sum_{i=1}^n \alpha_i - \frac{1}{2}\sum_{i=1}^n\sum_{j=1}^n\alpha_i\alpha_j\yi y^{(j)} k(\xi, \xv^{(j)})  \\
										& \text{s.t. } & 0 \le \alpha_i \le C, \forall\, i \in \nset \quad \text{and} \quad  \sum_{i=1}^n \alpha_i \yi = 0
									\end{eqnarray*}	
							%	
								Kernel representation of separating hyperplane:
							%
									$$ \fx = \sumin \alpha_i \yi k(\xi, \xv)  + \theta_0$$
							%
							\end{myblock}
							%
							%-------------------------------------------------------------------------------
							%   
							%-------------------------------------------------------------------------------
							\begin{myblock}{Gaussian Processes}
							%
								Bayesian Linear Model:
							%
								\begin{eqnarray*}
									\yi &=& \fxi + \epsi = \thetab^T \xi + \epsi, \quad \text{for } i \in \{1, \ldots, n\}
								\end{eqnarray*}
								%
								where $\epsi \sim \mathcal{N}(0, \sigma^2).$
							%
								Parameter vector $\thetab$ is stochastic and follows a distribution.\\
							%	
								
								Gaussian variant: 
							%	
								\begin{itemize}
									\setlength{\itemindent}{+.3in}
									\item Prior distribution: $\thetab \sim \mathcal{N}(\zero, \tau^2 \id_p)$ 
									\item Posterior distribution:	$
									\thetab ~|~ \Xmat, \yv \sim \mathcal{N}(\sigma^{-2}\bm{A}^{-1}\Xmat^\top\yv, \bm{A}^{-1})
									$ with $\bm{A}:= \sigma^{-2}\Xmat^\top\Xmat + \frac{1}{\tau^2} \id_p$
									\item Predictive distribution of $y_* = 	\thetab^\top \xv_*$ for a new observations $\xv_*$: 
									$$
									y_* ~|~ \Xmat, \yv, \xv_* \sim \mathcal{N}(\sigma^{-2}\yv^\top \Xmat \Amat^{-1}\xv_*, \xv_*^\top\Amat^{-1}\xv_*)
									$$
								\end{itemize}

							%	
							
								
							%
								\fbox{
									\parbox{\dimexpr\textwidth-2\fboxsep-2\fboxrule}{
										\begin{table}
											\begin{tabular}{cc}
												\textbf{Weight-Space View} & \textbf{Function-Space View} \vspace{4mm}\\ 
												Parameterize functions & \vspace{1mm}\\
												\footnotesize Example: $\fxt = \thetab^\top \xv$ & \vspace{3mm}\\
												Define distributions on $\thetab$ & Define distributions on $f$ \vspace{4mm}\\
												Inference in parameter space $\Theta$ & Inference in function space $\Hspace$
											\end{tabular}
										\end{table}  
									}
								}\\
							%


								$\hphantom{text}$\\

								\textbf{Gaussian Processes:} A function $\fx$ is generated by a GP $\gp$ if for \textbf{any finite} set of inputs $\left\{\xv^{(1)}, \dots, \xv^{(n)}\right\}$, the associated vector of function values $\bm{f} = \left(f(\xv^{(1)}), \dots, f(\xv^{(n)})\right)$ has a Gaussian distribution
								%
								$$
								\bm{f} = \left[f\left(\xi[1]\right),\dots, f\left(\xi[n]\right)\right] \sim \mathcal{N}\left(\bm{m}, \bm{K}\right),
								$$
								%
								with 
								%
								\begin{eqnarray*}
									\textbf{m} &:=& \left(m\left(\xi\right)\right)_{i}, \quad
									\textbf{K} := \left(k\left(\xi, \xv^{(j)}\right)\right)_{i,j}, 
								\end{eqnarray*}
								%
								where $m(\xv)$ is the mean function and $k(\xv, \xv^\prime)$ is the covariance function. 
							
							\end{myblock}\vfill
		% End First Column
		%%%%%%%%%%%%%%%%%%%%%%%%%%%%%%%%%%%%%%%%%%%%%%%%%%%%%%%%%%%%%%%%%%%%%%%%%%%%%%%%
						}
					\end{minipage}
				\end{beamercolorbox}
			\end{column}
			\begin{column}{.31\textwidth}
				\begin{beamercolorbox}[center]{postercolumn}
					\begin{minipage}{.98\textwidth}
						\parbox[t][\columnheight]{\textwidth}{
							%%%%%%%%%%%%%%%%%%%%%%%%%%%%%%%%%%%%%%%%%%%%%%%%%%%%%%%%%%%%%%%%%%%%%%%%%%%%%%%%
							% Begin Second Column
							\begin{myblock}{}  
							%
								Types of covariance functions:
								%		
								\begin{itemize}
									\setlength{\itemindent}{+.3in}
									%			
									\item $k(.,.)$ is stationary if it is as a function of $\bm{d} = \bm{x} - \bm{x}^\prime$, $ \leadsto k(\bm{d})$
									\item $k(.,.)$ is isotropic if it is a function of $r = \|\bm{x} - \bm{x}^\prime\|$,  $ \leadsto k(r)$
									\item $k(., .)$ is a dot product covariance function if $k$ is a function of $\bm{x}^T \bm{x}^\prime$
								\end{itemize}
							%
								Commonly used covariance functions
								
								\begin{center}
										\fbox{  
										\begin{tabular}{|c|c|}
											\hline
											Name & $k(\bm{x}, \bm{x}^\prime)$\\
											\hline
											constant & $\sigma_0^2$ \\ [1em]
											linear & $\sigma_0^2 + \bm{x}^T\bm{x}^\prime$ \\ [1em]
											polynomial & $(\sigma_0^2 + \bm{x}^T\bm{x}^\prime)^p$ \\ [1em]
											squared exponential & $\exp(- \frac{\|\bm{x} - \bm{x}^\prime\|^2}{2\ls^2})$ \\ [1em]
											Matérn & \begin{footnotesize} $\frac{1}{2^\nu \Gamma(\nu)}\biggl(\frac{\sqrt{2 \nu}}{\ls}\|\bm{x} - \bm{x}^\prime\|\biggr)^{\nu} K_\nu\biggl(\frac{\sqrt{2 \nu}}{\ls}\|\bm{x} - \bm{x}^\prime\|\biggr)$\end{footnotesize}  \\ [1em]
											exponential & $\exp\left(- \frac{\|\bm{x} - \bm{x}^\prime\|}{\ls}\right)$ \\ [1em]
											\hline
									\end{tabular} }\\
								\end{center}
									


								%
								\textbf{Posterior process} 
								
								Assuming a zero-mean GP prior $\mathcal{GP}\left(\bm{0}, k(\xv, \xv^\prime)\right).$ 
							%	
								For $ f_* = f\left(\xv_*\right)$ on single unobserved test point $\xv_*$ 
							%	
								\begin{eqnarray*}
									f_* ~|~ \xv_*, \Xmat, \bm{f} \sim \mathcal{N}(\bm{k}_{*}^{T}\Kmat^{-1}\bm{f}, \bm{k}_{**} - \bm{k}_*^T \Kmat ^{-1}\bm{k}_*),
								\end{eqnarray*}
							%
								where, $\Kmat = \left(k\left(\xi, \xv^{(j)}\right)\right)_{i,j}$, $\bm{k}_* = \left[k\left(\xv_*, \xi[1]\right), ..., k\left(\xv_*, \xi[n]\right)\right]$ and $ \bm{k}_{**}\ = k(\xv_*, \xv_*)$. \\
								
								%
								For multiple unobserved test points
								$
								\bm{f}_* = \left[f\left(\xi[1]_*\right), ..., f\left(\xi[m]_*\right)\right]:
								$
								\begin{eqnarray*}
									\bm{f}_* ~|~ \Xmat_*, \Xmat, \bm{f} \sim \mathcal{N}(\Kmat_{*}^{T}\Kmat^{-1}\bm{f}, \Kmat_{**} - \Kmat_*^T \Kmat ^{-1}\Kmat_*).
								\end{eqnarray*} 
								with $\Kmat_* = \left(k\left(\xi, \xv_*^{(j)}\right)\right)_{i,j}$, $\Kmat_{**} = \left(k\left(\xi[i]_*, \xi[j]_*\right)\right)_{i,j}$.\\
								
								Predictive mean when assuming a non-zero mean GP prior $\gp$ with mean $m(\xv):$ 
								$$
								m(\Xmat_*) + \Kmat_*\Kmat^{-1}\left(\bm{y} - m(\Xmat)\right)
								$$
								Predictive variance remains unchanged. \\
								
								\textbf{Noisy posterior process:}
							%	
								Assuming a zero-mean GP prior $\mathcal{GP}\left(\bm{0}, k(\xv, \xv^\prime)\right):$ 
							%	
								\begin{eqnarray*}
									\bm{f}_* ~|~ \Xmat_*, \Xmat, \bm{y} \sim \mathcal{N}(\bm{m}_{\text{post}}, \bm{K}_\text{post}).
								\end{eqnarray*}
								with nugget $\sigma^2 $ and
							%	 
								\begin{eqnarray*}
									\bm{m}_{\text{post}} &=& \Kmat_{*}^{T} \left(\Kmat+ \sigma^2 \cdot \id\right)^{-1}\bm{y} \\
									\bm{K}_\text{post} &=& \Kmat_{**} - \Kmat_*^T \left(\Kmat  + \sigma^2 \cdot \id\right)^{-1}	\Kmat_*,	
								\end{eqnarray*} 
							%	
								
								Predictive mean when assuming a non-zero mean GP prior $\gp$ with mean $m(\xv):$ 
								$$
								m(\Xmat_*) + \Kmat_*(\Kmat +\sigma^2 \id)^{-1}\left(\bm{y} - m(\Xmat)\right)
								$$
								Predictive variance remains unchanged.
							%
							%
							\end{myblock}
		% End Second Column					
		%%%%%%%%%%%%%%%%%%%%%%%%%%%%%%%%%%%%%%%%%%%%%%%%%%%%%%%%%%%%%%%%%%%%%%%%%%%%%%%%
						}
					\end{minipage}
				\end{beamercolorbox}
			\end{column}
			\begin{column}{.31\textwidth}
				\begin{beamercolorbox}[center]{postercolumn}
					\begin{minipage}{.98\textwidth}
						\parbox[t][\columnheight]{\textwidth}{
						%%%%%%%%%%%%%%%%%%%%%%%%%%%%%%%%%%%%%%%%%%%%%%%%%%%%%%%%%%%%%%%%%%%%%%%%%%%%%%%%
						% Begin Third Column#


						%-------------------------------------------------------------------------------
						% Regression Losses 
						%------------------------------------------------------------------------------- 
						\begin{myblock}{Boosting}
						%	

							\begin{algorithm}[H]
								\begin{algorithmic}[1]
									\State Initialize observation weights: $w^{[1](i)} = \frac{1}{n} \quad \forall i \in \nset$
									\For {$m = 1 \to M$}
									\State Fit classifier to training data with weights $\wm$ and get $\blh$
									\State Calculate weighted in-sample misclassification rate
									$$
									\errm = \sumin \wmi \cdot \mathds{1}_{\{\yi \,\neq\, \blh(\xi)\}}
									$$
									\State Compute: $ \betamh = \frac{1}{2} \log \left( \frac{1 - \errm}{\errm}\right)$
									\State Set: $w^{[m+1](i)} = \wmi \cdot \exp\left(- \betamh \cdot
									\yi \cdot \blh(\xi)\right) $
									\State Normalize $w^{[m+1](i)}$ such that $\sumin w^{[m+1](i)} = 1$
									\EndFor
									\State Output: $\fxh = \sum_{m=1}^{M} \betamh \blh(\xv)$
								\end{algorithmic}
								\caption{AdaBoost}
							\end{algorithm}
							%

							$\hphantom{text}$\\

							\begin{algorithm}[H]
							%	\begin{footnotesize}
									\begin{center}
										\caption{Gradient Boosting Algorithm}
										\begin{algorithmic}[1]
											\State Initialize $\hat{f}^{[0]}(\xv) = \argmin_{\bm{\theta}} \sumin L(\yi, b(\xi, \bm{\theta}))$
											%\State Set the learning rate $\beta$ to a small constant value
											\For{$m = 1 \to M$}
											\State For all $i$: $\rmi = -\left[\pd{\Lxyi}{\fxi}\right]_{f=\fmdh}$
											\State Fit a regression base learner to the pseudo-residuals $\rmi$:
											\State $\thetamh = \argmin \limits_{\bm{\theta}} \sumin (\rmi - b(\xi, \bm{\theta}))^2$
											%\State Line search: $\betamh = \argmin_{\beta} \sumin L(\yi, \fmd(\xv) + \beta b(\xv, \thetamh))$
											\State Set $\betam$ to $\beta$ being a small constant value or via line search
											\State Update $\fmh(\xv) = \fmdh(\xv) + \betam b(\xv, \thetamh)$
											\EndFor
											\State Output $\fh(\xv) = \hat{f}^{[M]}(\xv)$
										\end{algorithmic}
									\end{center}
							%	\end{footnotesize}
							\end{algorithm} 

							$\hphantom{text}$\\

							\begin{algorithm}[H]
							%	\begin{footnotesize}
									\begin{center}
										\caption{Gradient Boosting for Multiclass}
										\begin{algorithmic}[1]
											\State Initialize $f_{k}^{[0]}(\xv) = 0,\ k = 1,\ldots,g$
											\For{$m = 1 \to M$}
											\State Set $\pik^{[m]}(\xv) = \frac{\exp(f_k^{[m]}(\xv))}{\sum_j \exp(f_j^{[m]}(\xv))}, k = 1,\ldots,g$
											\For{$k = 1 \to g$}
											\State For all $i$: Compute $\rmi_k = \mathds{1}_{\{\yi = k\}} - \pik^{[m]}(\xi)$
											\State Fit a regression base learner $\hat{b}^{[m]}_k$ to the pseudo-residuals $\rmi_k$
											\State Obtain $\betamh_k$ by constant learning rate or line-search
											\State Update $\hat{f}_k^{[m]} = \hat{f}_k^{[m-1]} + \betamh_k \hat{b}^{[m]}_k$
											\EndFor
											\EndFor
											\State Output $\hat{f}_1^{[M]}, \ldots, \hat{f}_g^{[M]}$
										\end{algorithmic}
									\end{center}
							%	\end{footnotesize}
							\end{algorithm}
							
						\end{myblock} 
		%
		%
		%-------------------------------------------------------------------------------
		% Classification 
		%------------------------------------------------------------------------------- 

		%\begin{myblock}{Classification}
		% 				    We want to assign new observations to known categories according to criteria learned from a training set.  
		%             \vspace*{1ex}
		%             

		%$y \in \Yspace = \gset : $ categorical output variable (label)\\ 

		%\textbf{Classification} usually means to construct $g$ \textbf{discriminant functions}:
		
		%$f_1(\xv), \ldots, \fgx$, so that we choose our class as \\ $h(\xv) = \argmax_{k \in \gset} \fkx$ \\

		%\textbf{Linear Classifier:} functions $\fkx$ can be specified as linear functions\\

		% \hspace*{1ex}\textbf{Note: }All linear classifiers can represent non-linear decision boundaries \hspace*{1ex}in our original input space if we include derived features. For example: \hspace*{1ex}higher order interactions, polynomials or other transformations of x in \hspace*{1ex}the model.

		%\textbf{Binary classification: }If only 2 classes ($\Yspace = \setzo$ or  $\Yspace = \setmp$) exist, we can use a single discriminant function $\fx = f_{1}(\xv) - f_{2}(\xv)$.  \\


		% \textbf{Generative approach }models $\pdfxyk$, usually by making some assumptions about the structure of these distributions and employs the Bayes theorem: 
		% $\pikx = \postk \propto \pdfxyk \pik$. \\ %It allows the computation of \hspace*{1ex}$\pikx$. \\
		% \textbf{Examples}: Linear discriminant analysis (LDA), Quadratic discriminant analysis (QDA), Naive Bayes\\
		% 
		% \textbf{Discriminant approach }tries to optimize the discriminant functions directly, usually via empirical risk minimization:\\ 
		% $ \fh = \argmin_{f \in \Hspace} \riske(f) = \argmin_{f \in \Hspace} \sumin \Lxyi.$\\
		% \textbf{Examples}: Logistic/softmax regression, kNN


		%\end{myblock}

		%-------------------------------------------------------------------------------
		% HRO - Components of Learning 
		%-------------------------------------------------------------------------------          
		%\begin{myblock}{Components of Learning}

		%\textbf{Learning = Hypothesis space + Risk + Optimization} \\
		%\phantom{\textbf{Learning}} \textbf{= }$ \Hspace + \risket + \argmin_{\thetab \in \Theta} 
		%\risket$

		% 
		% \textbf{Learning &= Hypothesis space &+ Risk  &+ Optimization} \\
		% &= $\Hspace &+ \risket &+ \argmin_{\thetab \in \Theta} \risket$
		% 
		% \textbf{Hypothesis space: } Defines (and restricts!) what kind of model $f$
		% can be learned from the data.
		% 
		% Examples: linear functions, decision trees
		% 
		% \vspace*{0.5ex}
		% 
		% \textbf{Risk: } Quantifies how well a model performs on a given
		% data set. This allows us to rank candidate models in order to choose the best one.
		% 
		% Examples: squared error, negative (log-)likelihood
		% 
		% \vspace*{0.5ex}
		% 
		% \textbf{Optimization: } Defines how to search for the best model, i.e., the model with the smallest {risk}, in the hypothesis space.
		% 
		% Examples: gradient descent, quadratic programming


		%\end{myblock}
		% End Third Column
		%%%%%%%%%%%%%%%%%%%%%%%%%%%%%%%%%%%%%%%%%%%%%%%%%%%%%%%%%%%%%%%%%%%%%%%%%%%%%%%%
					}
					\end{minipage}
				\end{beamercolorbox}
			\end{column}
		\end{columns}

	\end{frame}
\end{document}
