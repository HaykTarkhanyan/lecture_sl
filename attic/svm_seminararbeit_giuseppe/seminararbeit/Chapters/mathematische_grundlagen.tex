\chapter{Grundlagen}
\label{grundlagen}

%\section{Lagrange-Methode}
%\section{Datensituation
%\textbf{Ausgangslage}:

Bei der Klassifizierung durch Support Vector Machines geht man von $N$ Trainingsdaten $(\mathbf{x}_1^{\top}, y_1), \hdots, (\mathbf{x}_N^{\top}, y_N)$ aus, wobei

\begin{tabular}{lll}
&$\mathbf{x}_i^{\top} \in \mathbb{R}^p$ & ein Merkmalsvektor mit $p$ Variablen und\\
&$y_i \in \{-1,+1\}$ & die Klassenzugeh�rigkeit der $i$-ten Beobachtung
\end{tabular}

ist. Die vorliegende Seminararbeit beschr�nkt sich auf die bin�re Klassifizierung durch Support Vector Machines. Die grundlegende Idee besteht darin, die Daten durch eine Hyperebene in zwei Klassen aufzuteilen. Eine Hyperebene trennt dabei einen $p$-dimensionalen Variablenraum in zwei Halbr�ume und hat selbst die Dimension $(p-1)$. Die Gleichung einer Hyperebene hat die Form
\begin{equation} 
\label{hyperebene}
%\mathcal{H} = 
\{\mathbf{x} \in \mathbb{R}^p \; | \; \mathbf{w}^\top \mathbf{x}+ b = 0\}, 
\end{equation}
wobei $\mathbf{w} \in \mathbb{R}^p$ ein Vektor orthogonal zur Hyperebene und $b \in \mathbb{R}$ die Verschiebung (vom Ursprung) ist.
Beispielsweise sind Hyperebenen
\begin{itemize}
\item in einem eindimensionalen Variablenraum alle Mengen, die aus einem Punkt bestehen,
\item in einem zweidimensionalen Variablenraum alle Geraden und
\item in einem dreidimensionalen Variablenraum alle Ebenen.
\end{itemize}

% Im eindimensionalen Variablenraum ($p=1$) ist jede Mengen, die aus einem Punkt besteht eine Hyperebene.

Das Ziel ist hierbei mit Hilfe einer Entscheidungsfunktion $f: \mathbb{R}^p \rightarrow \{-1,+1\}$ neu hinzukommende Beobachtungen $\mathbf{x}_{neu}$ m�glichst fehlerfrei in die negative Klasse $y_{neu}=-1$ oder in die positive Klasse $y_{neu}=+1$ zuzuordnen. Die Entscheidungsfunktion wird auf Basis der Trainingsdaten so bestimmt, dass der Ausdruck $f(\mathbf{x}_i) = y_i$ 

\begin{itemize}
\item im Falle linear trennbarer Daten f�r alle Beobachtungen $i= 1, \hdots, N$ und %(vgl. Kapitel \ref{chap:trennbar}) und
\item im Falle nicht linear trennbarer Daten f�r m�glichst viele Beobachtungen $i$ %(vgl. Kapitel \ref{chap:nichttrennbar}) 
\end{itemize}

gilt (\citealp[vgl.][Kap. 7.1]{scholkopf2001learning}).