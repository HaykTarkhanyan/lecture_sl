\chapter{Zusammenfassung und Ausblick}

Wenn das prim�re Optimierungsproblem bekannt ist, lassen sich bei linear trennbaren Daten die Schritte zur Bestimmung der Hyperebenenenparameter $(\mathbf{w}, b)$ wie folgt zusammenfassen:

\begin{enumerate}
\item Lagrangefunktion aufstellen
\item Lagrange-duale Funktion und duales Optimierungsproblem herleiten
\item Lagrange-Multiplikator $\alpha_i$ durch duales Optimierungsproblem bestimmen
\item Richtungsvektor $\mathbf{w} = \sum_{i=1}^N \alpha_i y_i \mathbf{x}_i$ der kanonischen Hyperebene berechnen
\item Verschiebung $b = y_j - \sum_{i=1}^N \alpha_i y_i \mathbf{x}_i^\top \mathbf{x}_j  $ der Hyperebene ermitteln
\item Entscheidungsfunktion 
$f(\mathbf{x}_j) = \text{sgn} \left (\mathbf{w}^\top \mathbf{x}_j + b \right ) =  \text{sgn} \left (\sum_{i=1}^N \alpha_i y_i \mathbf{x}_i^\top \mathbf{x}_j  + b \right )$ aufstellen
\end{enumerate}

Bei nicht linear trennbaren Daten ist eine M�glichkeit die Verwendung des Kern Tricks, der
% mit Hilfe einer Kernfunktion $K$ den Merkmalsvektor $\mathbf{x}$
 die Daten in einem h�herdimensionalen Variablenraum �berf�hrt, in dem sie linear trennbar sind. Die Schritte zur Bestimmung der Hyperebenenenparameter �ndern sich darin, dass der Merkmalsvektor $\mathbf{x}_i$ f�r alle $i=1, \hdots, N$ durch einen h�herdimensionalen Merkmalsvektor $\Phi(\mathbf{x}_i)$ ersetzt wird.

Die andere M�glichkeit ist die Verwendung der Soft Margin Hyperebene. Hierbei werden durch Einf�hrung von Schlupfvariablen Fehlklassifizierungen erm�glicht, die
%$\xi_i$ die Nebenbedingung aus Gleichung \ref{eq:nb} so abgeschw�cht wird, dass Fehlklassifizierungen erlaubt 
zum Ausgleich im prim�ren Optimierungsproblem entsprechend bestraft werden. 
%In Kapitel \ref{sec:softmargin} wurde gezeigt, dass 
Obwohl das prim�re Optimierungsproblem anders als im linear trennbaren Fall ist, bleibt die Lagrange-duale Funktion dennoch dieselbe (vgl. Kapitel \ref{sec:softmargin}). Das duale Optimierungsproblem �ndert sich darin, dass die zus�tzliche Nebenbedingung $\alpha_i \leq C$ vorkommt. Die oben genannten Schritte k�nnen hier analog angewendet werden.

Zudem kann der Kern Trick auch bei der Soft Margin Hyperebene angewendet werden. Hierbei ersetzt man bei der Herleitung der Soft Margin Hyperebene den Merkmalsvektor $\mathbf{x}$ durch $\Phi(\mathbf{x})$.
%\begin{itemize}
%\item Erweiterung f�r Regressionsprobleme
%\item Erweiterung f�r mehrkategorialen Response, z.B. durch \textbf{Paarweise Klassifikation:}
%\begin{itemize}
%\item Bilde Klassifikatoren f�r jedes m�gliche Paar der $K$ Klassen
%\item Zuordung neuer Beobachtungen durch Mehrheitsentscheid in die Klasse $k \in \{1, \hdots, K \}$ 
%\end{itemize}
%\end{itemize}

In dieser Seminararbeit wurde nur die bin�re Klassifikation betrachtet. In der Literatur findet man Erweiterungen, die die Verwendung von Support Vector Machines auf mehrkategoriale Klassifikationsprobleme und Regressionprobleme erm�glichen. 

Ein Beispiel f�r den Umgang mit $K>2$ Klassen ist die paarweise Klassifikation. Hierbei werden Klassifikatoren f�r jedes m�gliche Paar der $K$ Klassen gebildet und die Beobachtungen anschlie�end durch Mehrheitsentscheid in die Klasse $k \in \{1, \hdots, K \}$ zugeordnet (\citealp[vgl.][Kap. 7.6]{hastie2011elements}).

F�r die Erweiterung auf Regressionsprobleme muss das Optimierungsproblem zun�chst in die \textbf{loss + penalty} Form gebracht werden.
Dies geschieht indem man die Definition der Schlupfvariablen
$$\xi_i = \max \{0, 1 - y_i ( \mathbf{w}^\top \mathbf{x}_i + b )\} = [1 - y_i \underbrace{( \mathbf{w}^\top \mathbf{x}_i + b )}_{=:f(\mathbf{x}_i)}]_{+}$$ %= [1 - y_i f(\mathbf{x}_i)]_{+}
in ein leicht abge�ndertes Optimierungsproblem aus Gleichung (\ref{eq:softmargin}) 
$$\min \hspace{3pt} \frac{1}{2} ||\mathbf{w}||^2 + C \sum_{i=1}^N \xi_i \; \Leftrightarrow \; \min \hspace{3pt} \frac{\lambda}{2} ||\mathbf{w}||^2 + \sum_{i=1}^N \xi_i  \text{ mit } \lambda = 1/C$$
einsetzt. Daraus ergibt sich folgende \textbf{loss + penalty} Form:
$$\min \hspace{3pt}  \sum_{i=1}^N[1-y_i f(\mathbf{x}_i)]_{+} + \frac{\lambda}{2} ||\mathbf{w}||^2$$
Durch Verwendung der Schlupfvariablen impliziert man als Verlustfunktion %$L(y_i, f(\mathbf{x}_i))$
den sogenannten \textbf{hinge loss}, der die Form $L(y_i, f(\mathbf{x}_i)) = [1 - y_i f(\mathbf{x}_i)]_{+}$ hat. Verwendet man eine andere Verlustfunktion $L(y_i, f(\mathbf{x}_i))$, k�nnen Support Vector Machines auch auf Regressionsprobleme angewendet werden (\citealp[vgl.][Kap. 5.1]{gunn1998support}; \citealp[Kap. 12.3.2]{hastie2011elements}).