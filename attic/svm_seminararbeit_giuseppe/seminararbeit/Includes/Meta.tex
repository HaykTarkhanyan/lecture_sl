% Meta-Informationen -----------------------------------------------------------
%   Definition von globalen Parametern, die im gesamten Dokument verwendet
%   werden k�nnen (z.B auf dem Deckblatt etc.).
%
%   ACHTUNG: Wenn die Texte Umlaute oder ein Esszet enthalten, muss der folgende
%            Befehl bereits an dieser Stelle aktiviert werden:
            \usepackage[latin1]{inputenc}
% ------------------------------------------------------------------------------
\newcommand{\titel}{Support Vector Machines}
\newcommand{\untertitel}{}
\newcommand{\art}{Seminar}
\newcommand{\arttitel}{--Regularisierungstechniken und strukturierte Regression--}
%\newcommand{\fachgebiet}{Software-Engineering}
\newcommand{\autor}{Giuseppe Casalicchio}
%\newcommand{\studienbereich}{Software-Engineering}
\newcommand{\matrikelnr}{8049501}
\newcommand{\erstgutachter}{Prof. Dr. Gerhard Tutz}
\newcommand{\zweitgutachter}{Wolfgang P��necker}
\newcommand{\jahr}{8. M�rz 2013}
\newcommand{\efp}{\text{\textit{efp}}}
%\newcommand{\ort}{Berlin}
%\newcommand{\logo}{lmu_siegel.pdf}




%%\usepackage[utf8]{inputenc}
%\usepackage{german}

%%% Zaehler fuer Aufgaben
\newcounter{ex}
\setcounter{ex}{0}
\newcommand{\ex}{\item{\stepcounter{ex} \textbf{Aufgabe
                       \arabic{ex} }}\newline}


%%% Theoremumgebungen
\usepackage{amsmath}
\usepackage{amssymb}
\usepackage{amsthm}

   \newtheoremstyle{ththm}% name
     {\topsep}%      Space above
     {\topsep}%      Space below
     {\normalfont}%         Body font
     {}%         Indent amount (empty = no indent, \parindent = para indent)
     {\normalfont\bfseries}% Thm head font
     {}%        Punctuation after thm head
     {\newline}%     Space after thm head: " " = normal interword space;
           %       \newline = linebreak
     {\thmname{#1}\thmnumber{ #2}\thmnote{ (#3)}}%         Thm head spec (can be left empty, meaning `normal')



%% Definition
\theoremstyle{ththm}
\newtheorem{defi}{Definition}[chapter]
\newtheorem{satz}[defi]{Satz}
\newtheorem{lemma}[defi]{Lemma}
\newtheorem{korollar}[defi]{Korollar}
\newtheorem{beispiel}{Beispiel}[chapter]

\theoremstyle{remark}
\newtheorem*{beme}{Bemerkung}

\newenvironment{beweis}{\begin{proof}[Beweis] \textcolor{white}{weiss} \newline}{\end{proof}}
